%% Generated by Sphinx.
\def\sphinxdocclass{report}
\documentclass[letterpaper,10pt,english]{sphinxmanual}
\ifdefined\pdfpxdimen
   \let\sphinxpxdimen\pdfpxdimen\else\newdimen\sphinxpxdimen
\fi \sphinxpxdimen=.75bp\relax
\ifdefined\pdfimageresolution
    \pdfimageresolution= \numexpr \dimexpr1in\relax/\sphinxpxdimen\relax
\fi
%% let collapsible pdf bookmarks panel have high depth per default
\PassOptionsToPackage{bookmarksdepth=5}{hyperref}

\PassOptionsToPackage{booktabs}{sphinx}
\PassOptionsToPackage{colorrows}{sphinx}

\PassOptionsToPackage{warn}{textcomp}
\usepackage[utf8]{inputenc}
\ifdefined\DeclareUnicodeCharacter
% support both utf8 and utf8x syntaxes
  \ifdefined\DeclareUnicodeCharacterAsOptional
    \def\sphinxDUC#1{\DeclareUnicodeCharacter{"#1}}
  \else
    \let\sphinxDUC\DeclareUnicodeCharacter
  \fi
  \sphinxDUC{00A0}{\nobreakspace}
  \sphinxDUC{2500}{\sphinxunichar{2500}}
  \sphinxDUC{2502}{\sphinxunichar{2502}}
  \sphinxDUC{2514}{\sphinxunichar{2514}}
  \sphinxDUC{251C}{\sphinxunichar{251C}}
  \sphinxDUC{2572}{\textbackslash}
\fi
\usepackage{cmap}
\usepackage[T1]{fontenc}
\usepackage{amsmath,amssymb,amstext}
\usepackage{babel}



\usepackage{tgtermes}
\usepackage{tgheros}
\renewcommand{\ttdefault}{txtt}



\usepackage[Bjarne]{fncychap}
\usepackage{sphinx}

\fvset{fontsize=auto}
\usepackage{geometry}


% Include hyperref last.
\usepackage{hyperref}
% Fix anchor placement for figures with captions.
\usepackage{hypcap}% it must be loaded after hyperref.
% Set up styles of URL: it should be placed after hyperref.
\urlstyle{same}

\addto\captionsenglish{\renewcommand{\contentsname}{Contents:}}

\usepackage{sphinxmessages}
\setcounter{tocdepth}{1}



\title{CoolDwarf}
\date{Jun 12, 2024}
\release{}
\author{Emily M.\@{} Boudreaux}
\newcommand{\sphinxlogo}{\vbox{}}
\renewcommand{\releasename}{}
\makeindex
\begin{document}

\ifdefined\shorthandoff
  \ifnum\catcode`\=\string=\active\shorthandoff{=}\fi
  \ifnum\catcode`\"=\active\shorthandoff{"}\fi
\fi

\pagestyle{empty}
\sphinxmaketitle
\pagestyle{plain}
\sphinxtableofcontents
\pagestyle{normal}
\phantomsection\label{\detokenize{index::doc}}


\sphinxAtStartPar
Welcome to the CoolDwarf project. This project aims to provide a easy to use and physically robsut
3D cooling model for fully convective stars.

\sphinxAtStartPar
CoolDwarf is in very early stages of development and is not yet ready for general or scientific use. However,
we welcome any feedback or contributions to the project.

\sphinxAtStartPar
The CoolDwarf project is hosted on GitHub at \sphinxurl{https://github.com/tboudreaux/CoolDwarf}


\chapter{Depedenices}
\label{\detokenize{index:depedenices}}
\sphinxAtStartPar
CoolDwarf requires the following packages to be installed:
\begin{itemize}
\item {} 
\sphinxAtStartPar
numpy

\item {} 
\sphinxAtStartPar
scipy

\item {} 
\sphinxAtStartPar
matplotlib

\item {} 
\sphinxAtStartPar
tqdm

\item {} 
\sphinxAtStartPar
torch

\end{itemize}

\sphinxAtStartPar
Optional packages:
\begin{itemize}
\item {} 
\sphinxAtStartPar
cupy

\end{itemize}

\sphinxAtStartPar
If you are using a CUDA enabled GPU, you can install the cupy package to speed up the calculations
significantly. If you do not have a CUDA enabled GPU, you can still use the package, but it will be
significantly slower. CoolDwarf will automatically detect if cupy is installed and use it if it is available.


\chapter{Installation}
\label{\detokenize{index:installation}}
\sphinxAtStartPar
To install the CoolDwarf package, you can use the following command:

\begin{sphinxVerbatim}[commandchars=\\\{\}]
git\PYG{+w}{ }clone\PYG{+w}{ }https://github.com/tboudreaux/CoolDwarf.git
\PYG{n+nb}{cd}\PYG{+w}{ }CoolDwarf
pip\PYG{+w}{ }install\PYG{+w}{ }.
\end{sphinxVerbatim}


\chapter{Usage}
\label{\detokenize{index:usage}}
\sphinxAtStartPar
The CoolDwarf package is designed to be easy to use. The primary entry point for
using the package is the CoolDwarf.star.VoxelSphere class. This class is used to create a 5D
model of a star.

\sphinxAtStartPar
The model is contructed of a grid of equal volume elements spread over a sphere. A MESA model
is used to provide the initial temperature and density profiles of the star. An equation of state
from Chabrier and Debras 2021 is used to calculate the pressure and energy of the star. The radiative opacity
of the star is currentl treated monocromatically and with a simplistic Kramer’s opacity model (this is a high
priority area for improvement).

\sphinxAtStartPar
Evolution of the Cooling model is preformed by calculating the radiative and convective energy gradients
at each grid point and to find the new energy after some small time step. The Equation of state is then inverted
to update the density, temperature, and pressure of the model.

\sphinxAtStartPar
Timesteps are dynamicall calculated using the Courant\sphinxhyphen{}Friedrichs\sphinxhyphen{}Lewy (CFL) condition. The CFL condition is used to ensure
that the timestep is small enough to prevent the model from becoming unstable. If the user wishes to use a fixed timestep,
they can set the cfl\_factor to infinity. Alternativley the timestep can be fixed by setting it lower than the CFL condition.

\sphinxAtStartPar
Currently, numerical instabilities exist; however, we are working to resolve these issues. A breif example
of how to use the package is shown below (note that neither the EOS tables nor the MESA model are included in the
repository; however, these are either readily available online (in the case of the EOS model) or can be generated
easily using MESA):

\begin{sphinxVerbatim}[commandchars=\\\{\}]
\PYG{k+kn}{from} \PYG{n+nn}{CoolDwarf}\PYG{n+nn}{.}\PYG{n+nn}{star} \PYG{k+kn}{import} \PYG{n}{VoxelSphere}\PYG{p}{,} \PYG{n}{default\PYGZus{}tol}
\PYG{k+kn}{from} \PYG{n+nn}{CoolDwarf}\PYG{n+nn}{.}\PYG{n+nn}{utils} \PYG{k+kn}{import} \PYG{n}{setup\PYGZus{}logging}
\PYG{k+kn}{from} \PYG{n+nn}{CoolDwarf}\PYG{n+nn}{.}\PYG{n+nn}{EOS} \PYG{k+kn}{import} \PYG{n}{get\PYGZus{}eos}
\PYG{k+kn}{from} \PYG{n+nn}{CoolDwarf}\PYG{n+nn}{.}\PYG{n+nn}{opac} \PYG{k+kn}{import} \PYG{n}{KramerOpac}
\PYG{k+kn}{from} \PYG{n+nn}{CoolDwarf}\PYG{n+nn}{.}\PYG{n+nn}{utils}\PYG{n+nn}{.}\PYG{n+nn}{output} \PYG{k+kn}{import} \PYG{n}{binmod}

\PYG{n}{modelWriter} \PYG{o}{=} \PYG{n}{binmod}\PYG{p}{(}\PYG{p}{)}

\PYG{n}{setup\PYGZus{}logging}\PYG{p}{(}\PYG{n}{debug}\PYG{o}{=}\PYG{k+kc}{False}\PYG{p}{)}

\PYG{n}{EOS} \PYG{o}{=} \PYG{n}{get\PYGZus{}eos}\PYG{p}{(}\PYG{l+s+s2}{\PYGZdq{}}\PYG{l+s+s2}{EOS/TABLEEOS\PYGZus{}2021\PYGZus{}Trho\PYGZus{}Y0292\PYGZus{}v1}\PYG{l+s+s2}{\PYGZdq{}}\PYG{p}{,} \PYG{l+s+s2}{\PYGZdq{}}\PYG{l+s+s2}{CD21}\PYG{l+s+s2}{\PYGZdq{}}\PYG{p}{)}
\PYG{n}{opac} \PYG{o}{=} \PYG{n}{KramerOpac}\PYG{p}{(}\PYG{l+m+mf}{0.7}\PYG{p}{,} \PYG{l+m+mf}{0.02}\PYG{p}{)}
\PYG{n}{sphere} \PYG{o}{=} \PYG{n}{VoxelSphere}\PYG{p}{(}
   \PYG{l+m+mf}{8e31}\PYG{p}{,}
   \PYG{l+s+s2}{\PYGZdq{}}\PYG{l+s+s2}{BrownDwarfMESA/BD\PYGZus{}TEST.mod}\PYG{l+s+s2}{\PYGZdq{}}\PYG{p}{,}
   \PYG{n}{EOS}\PYG{p}{,}
   \PYG{n}{opac}\PYG{p}{,}
   \PYG{n}{radialResolution}\PYG{o}{=}\PYG{l+m+mi}{100}\PYG{p}{,}
   \PYG{n}{altitudinalResolition}\PYG{o}{=}\PYG{l+m+mi}{100}\PYG{p}{,}
   \PYG{n}{azimuthalResolition}\PYG{o}{=}\PYG{l+m+mi}{100}\PYG{p}{,}
   \PYG{n}{cfl\PYGZus{}factor} \PYG{o}{=} \PYG{l+m+mf}{0.4}\PYG{p}{,}
\PYG{p}{)}
\PYG{n}{sphere}\PYG{o}{.}\PYG{n}{evolve}\PYG{p}{(}\PYG{n}{maxTime} \PYG{o}{=} \PYG{l+m+mi}{60}\PYG{o}{*}\PYG{l+m+mi}{60}\PYG{o}{*}\PYG{l+m+mi}{24}\PYG{p}{,} \PYG{n}{pbar}\PYG{o}{=}\PYG{k+kc}{False}\PYG{p}{)}
\end{sphinxVerbatim}

\sphinxstepscope


\section{CoolDwarf package}
\label{\detokenize{CoolDwarf:cooldwarf-package}}\label{\detokenize{CoolDwarf::doc}}

\subsection{Subpackages}
\label{\detokenize{CoolDwarf:subpackages}}
\sphinxstepscope


\subsubsection{CoolDwarf.EOS package}
\label{\detokenize{CoolDwarf.EOS:cooldwarf-eos-package}}\label{\detokenize{CoolDwarf.EOS::doc}}

\paragraph{Subpackages}
\label{\detokenize{CoolDwarf.EOS:subpackages}}
\sphinxstepscope


\subparagraph{CoolDwarf.EOS.ChabrierDebras2021 package}
\label{\detokenize{CoolDwarf.EOS.ChabrierDebras2021:cooldwarf-eos-chabrierdebras2021-package}}\label{\detokenize{CoolDwarf.EOS.ChabrierDebras2021::doc}}

\subparagraph{Submodules}
\label{\detokenize{CoolDwarf.EOS.ChabrierDebras2021:submodules}}

\subparagraph{CoolDwarf.EOS.ChabrierDebras2021.EOS module}
\label{\detokenize{CoolDwarf.EOS.ChabrierDebras2021:module-CoolDwarf.EOS.ChabrierDebras2021.EOS}}\label{\detokenize{CoolDwarf.EOS.ChabrierDebras2021:cooldwarf-eos-chabrierdebras2021-eos-module}}\index{module@\spxentry{module}!CoolDwarf.EOS.ChabrierDebras2021.EOS@\spxentry{CoolDwarf.EOS.ChabrierDebras2021.EOS}}\index{CoolDwarf.EOS.ChabrierDebras2021.EOS@\spxentry{CoolDwarf.EOS.ChabrierDebras2021.EOS}!module@\spxentry{module}}
\sphinxAtStartPar
EOS.py \textendash{} EOS class for Chabrier Debras 2021 EOS tables

\sphinxAtStartPar
This module contains the EOS class for the Chabrier Debras 2021 EOS tables. The class is designed to be used with the
CoolDwarf Stellar Structure code, and provides the necessary functions to interpolate the EOS tables.

\sphinxAtStartPar
As with all EOS classes in CoolDwarf, the CH21EOS class is designed to accept linear values in cgs units, and return
linear values in cgs units.


\subparagraph{Dependencies}
\label{\detokenize{CoolDwarf.EOS.ChabrierDebras2021:dependencies}}\begin{itemize}
\item {} 
\sphinxAtStartPar
pandas

\item {} 
\sphinxAtStartPar
scipy

\item {} 
\sphinxAtStartPar
cupy

\item {} 
\sphinxAtStartPar
torch

\end{itemize}


\subparagraph{Example usage}
\label{\detokenize{CoolDwarf.EOS.ChabrierDebras2021:example-usage}}
\begin{sphinxVerbatim}[commandchars=\\\{\}]
\PYG{g+gp}{\PYGZgt{}\PYGZgt{}\PYGZgt{} }\PYG{k+kn}{from} \PYG{n+nn}{CoolDwarf}\PYG{n+nn}{.}\PYG{n+nn}{EOS}\PYG{n+nn}{.}\PYG{n+nn}{ChabrierDebras2021}\PYG{n+nn}{.}\PYG{n+nn}{EOS} \PYG{k+kn}{import} \PYG{n}{CH21EOS}
\PYG{g+gp}{\PYGZgt{}\PYGZgt{}\PYGZgt{} }\PYG{n}{eos} \PYG{o}{=} \PYG{n}{CH21EOS}\PYG{p}{(}\PYG{l+s+s2}{\PYGZdq{}}\PYG{l+s+s2}{path/to/eos/table}\PYG{l+s+s2}{\PYGZdq{}}\PYG{p}{)}
\PYG{g+gp}{\PYGZgt{}\PYGZgt{}\PYGZgt{} }\PYG{n}{pressure} \PYG{o}{=} \PYG{n}{eos}\PYG{o}{.}\PYG{n}{pressure}\PYG{p}{(}\PYG{l+m+mf}{7.0}\PYG{p}{,} \PYG{o}{\PYGZhy{}}\PYG{l+m+mf}{2.0}\PYG{p}{)}
\PYG{g+gp}{\PYGZgt{}\PYGZgt{}\PYGZgt{} }\PYG{n}{energy} \PYG{o}{=} \PYG{n}{eos}\PYG{o}{.}\PYG{n}{energy}\PYG{p}{(}\PYG{l+m+mf}{7.0}\PYG{p}{,} \PYG{o}{\PYGZhy{}}\PYG{l+m+mf}{2.0}\PYG{p}{)}
\end{sphinxVerbatim}
\index{CH21EOS (class in CoolDwarf.EOS.ChabrierDebras2021.EOS)@\spxentry{CH21EOS}\spxextra{class in CoolDwarf.EOS.ChabrierDebras2021.EOS}}

\begin{fulllineitems}
\phantomsection\label{\detokenize{CoolDwarf.EOS.ChabrierDebras2021:CoolDwarf.EOS.ChabrierDebras2021.EOS.CH21EOS}}
\pysigstartsignatures
\pysiglinewithargsret{\sphinxbfcode{\sphinxupquote{class\DUrole{w}{ }}}\sphinxcode{\sphinxupquote{CoolDwarf.EOS.ChabrierDebras2021.EOS.}}\sphinxbfcode{\sphinxupquote{CH21EOS}}}{\sphinxparam{\DUrole{n}{tablePath}}}{}
\pysigstopsignatures
\sphinxAtStartPar
Bases: \sphinxcode{\sphinxupquote{object}}

\sphinxAtStartPar
CH21EOS \textendash{} EOS class for Chabrier Debras 2021 EOS tables

\sphinxAtStartPar
This class is designed to be used with the CoolDwarf Stellar Structure code, and provides the necessary functions
to interpolate the Chabrier Debras 2021 EOS tables.
\begin{quote}\begin{description}
\sphinxlineitem{Parameters}\begin{description}
\sphinxlineitem{\sphinxstylestrong{tablePath}}{[}str{]}
\sphinxAtStartPar
Path to the Chabrier Debras 2021 EOS table

\end{description}

\sphinxlineitem{Attributes}\begin{description}
\sphinxlineitem{{\hyperref[\detokenize{CoolDwarf.EOS.ChabrierDebras2021:CoolDwarf.EOS.ChabrierDebras2021.EOS.CH21EOS.TRange}]{\sphinxcrossref{\sphinxcode{\sphinxupquote{TRange}}}}}}{[}tuple{]}
\sphinxAtStartPar
Tuple containing the minimum and maximum temperature (log10(T)) in the EOS table

\sphinxlineitem{{\hyperref[\detokenize{CoolDwarf.EOS.ChabrierDebras2021:CoolDwarf.EOS.ChabrierDebras2021.EOS.CH21EOS.rhoRange}]{\sphinxcrossref{\sphinxcode{\sphinxupquote{rhoRange}}}}}}{[}tuple{]}
\sphinxAtStartPar
Tuple containing the minimum and maximum density (log10(ρ)) in the EOS table

\end{description}

\end{description}\end{quote}
\subsubsection*{Methods}


\begin{savenotes}\sphinxattablestart
\sphinxthistablewithglobalstyle
\centering
\begin{tabulary}{\linewidth}[t]{TT}
\sphinxtoprule
\sphinxtableatstartofbodyhook
\sphinxAtStartPar
\sphinxstylestrong{pressure(logT, logRho)}
&
\sphinxAtStartPar
Interpolates the pressure at the given temperature and density
\\
\sphinxhline
\sphinxAtStartPar
\sphinxstylestrong{energy(logT, logRho)}
&
\sphinxAtStartPar
Interpolates the energy at the given temperature and density
\\
\sphinxbottomrule
\end{tabulary}
\sphinxtableafterendhook\par
\sphinxattableend\end{savenotes}
\index{TRange (CoolDwarf.EOS.ChabrierDebras2021.EOS.CH21EOS property)@\spxentry{TRange}\spxextra{CoolDwarf.EOS.ChabrierDebras2021.EOS.CH21EOS property}}

\begin{fulllineitems}
\phantomsection\label{\detokenize{CoolDwarf.EOS.ChabrierDebras2021:CoolDwarf.EOS.ChabrierDebras2021.EOS.CH21EOS.TRange}}
\pysigstartsignatures
\pysigline{\sphinxbfcode{\sphinxupquote{property\DUrole{w}{ }}}\sphinxbfcode{\sphinxupquote{TRange}}}
\pysigstopsignatures
\sphinxAtStartPar
Tuple containing the minimum and maximum temperature (log10(T)) in the EOS table
\begin{quote}\begin{description}
\sphinxlineitem{Returns}\begin{description}
\sphinxlineitem{tuple}
\sphinxAtStartPar
Tuple containing the minimum and maximum temperature (log10(T)) in the EOS table

\end{description}

\end{description}\end{quote}

\end{fulllineitems}

\index{check\_forward\_params() (CoolDwarf.EOS.ChabrierDebras2021.EOS.CH21EOS method)@\spxentry{check\_forward\_params()}\spxextra{CoolDwarf.EOS.ChabrierDebras2021.EOS.CH21EOS method}}

\begin{fulllineitems}
\phantomsection\label{\detokenize{CoolDwarf.EOS.ChabrierDebras2021:CoolDwarf.EOS.ChabrierDebras2021.EOS.CH21EOS.check_forward_params}}
\pysigstartsignatures
\pysiglinewithargsret{\sphinxbfcode{\sphinxupquote{check\_forward\_params}}}{\sphinxparam{\DUrole{n}{logT}}\sphinxparamcomma \sphinxparam{\DUrole{n}{logRho}}}{}
\pysigstopsignatures
\sphinxAtStartPar
Check if the given temperature and density are within the bounds of the EOS table. If the values are out of
bounds, a ValueError is raised.
\begin{quote}\begin{description}
\sphinxlineitem{Parameters}\begin{description}
\sphinxlineitem{\sphinxstylestrong{logT}}{[}float{]}
\sphinxAtStartPar
Log10 of the temperature in K

\sphinxlineitem{\sphinxstylestrong{logRho}}{[}float{]}
\sphinxAtStartPar
Log10 of the density in g/cm\textasciicircum{}3

\end{description}

\sphinxlineitem{Raises}\begin{description}
\sphinxlineitem{ValueError}
\sphinxAtStartPar
If the temperature or density is out of bounds

\end{description}

\end{description}\end{quote}

\end{fulllineitems}

\index{energy() (CoolDwarf.EOS.ChabrierDebras2021.EOS.CH21EOS method)@\spxentry{energy()}\spxextra{CoolDwarf.EOS.ChabrierDebras2021.EOS.CH21EOS method}}

\begin{fulllineitems}
\phantomsection\label{\detokenize{CoolDwarf.EOS.ChabrierDebras2021:CoolDwarf.EOS.ChabrierDebras2021.EOS.CH21EOS.energy}}
\pysigstartsignatures
\pysiglinewithargsret{\sphinxbfcode{\sphinxupquote{energy}}}{\sphinxparam{\DUrole{n}{logT}}\sphinxparamcomma \sphinxparam{\DUrole{n}{logRho}}}{}
\pysigstopsignatures
\sphinxAtStartPar
Find the energy at the given temperature and density.
\begin{quote}\begin{description}
\sphinxlineitem{Parameters}\begin{description}
\sphinxlineitem{\sphinxstylestrong{logT}}{[}float{]}
\sphinxAtStartPar
Log10 of the temperature in K

\sphinxlineitem{\sphinxstylestrong{logRho}}{[}float{]}
\sphinxAtStartPar
Log10 of the density in g/cm\textasciicircum{}3

\end{description}

\end{description}\end{quote}

\end{fulllineitems}

\index{energy\_torch() (CoolDwarf.EOS.ChabrierDebras2021.EOS.CH21EOS method)@\spxentry{energy\_torch()}\spxextra{CoolDwarf.EOS.ChabrierDebras2021.EOS.CH21EOS method}}

\begin{fulllineitems}
\phantomsection\label{\detokenize{CoolDwarf.EOS.ChabrierDebras2021:CoolDwarf.EOS.ChabrierDebras2021.EOS.CH21EOS.energy_torch}}
\pysigstartsignatures
\pysiglinewithargsret{\sphinxbfcode{\sphinxupquote{energy\_torch}}}{\sphinxparam{\DUrole{n}{logT}}\sphinxparamcomma \sphinxparam{\DUrole{n}{logRho}}}{}
\pysigstopsignatures
\sphinxAtStartPar
Find the energy at the given temperature and density using PyTorch tensors.
\begin{quote}\begin{description}
\sphinxlineitem{Parameters}\begin{description}
\sphinxlineitem{\sphinxstylestrong{logT}}{[}torch.Tensor{]}
\sphinxAtStartPar
Log10 of the temperature in K

\sphinxlineitem{\sphinxstylestrong{logRho}}{[}torch.Tensor{]}
\sphinxAtStartPar
Log10 of the density in g/cm\textasciicircum{}3

\end{description}

\end{description}\end{quote}

\end{fulllineitems}

\index{parse\_table() (CoolDwarf.EOS.ChabrierDebras2021.EOS.CH21EOS method)@\spxentry{parse\_table()}\spxextra{CoolDwarf.EOS.ChabrierDebras2021.EOS.CH21EOS method}}

\begin{fulllineitems}
\phantomsection\label{\detokenize{CoolDwarf.EOS.ChabrierDebras2021:CoolDwarf.EOS.ChabrierDebras2021.EOS.CH21EOS.parse_table}}
\pysigstartsignatures
\pysiglinewithargsret{\sphinxbfcode{\sphinxupquote{parse\_table}}}{}{}
\pysigstopsignatures
\sphinxAtStartPar
Parse the Chabrier Debras 2021 EOS table and store the data in the class. Parsing is done using regular
expressions to extract the data from the table file. Columns are named according to the table format, and the
data is stored in a pandas DataFrame. The temperature and density values are extracted and stored in separate
arrays, and the pressure and energy values are interpolated using RegularGridInterpolator.

\sphinxAtStartPar
Column Names (in order):
\sphinxhyphen{} logT
\sphinxhyphen{} logP
\sphinxhyphen{} logRho
\sphinxhyphen{} logU
\sphinxhyphen{} logS
\sphinxhyphen{} dlrho/dlT\_P
\sphinxhyphen{} dlrho/dlP\_T
\sphinxhyphen{} dlS/dlT\_P
\sphinxhyphen{} dlS/dlP\_T

\sphinxAtStartPar
The EOS table is stored as a 3D array, with the first dimension corresponding to the temperature, the second
dimension corresponding to the density, and the third dimension corresponding to the columns.

\sphinxAtStartPar
The temperature and density arrays are stored as 1D arrays.

\end{fulllineitems}

\index{pressure() (CoolDwarf.EOS.ChabrierDebras2021.EOS.CH21EOS method)@\spxentry{pressure()}\spxextra{CoolDwarf.EOS.ChabrierDebras2021.EOS.CH21EOS method}}

\begin{fulllineitems}
\phantomsection\label{\detokenize{CoolDwarf.EOS.ChabrierDebras2021:CoolDwarf.EOS.ChabrierDebras2021.EOS.CH21EOS.pressure}}
\pysigstartsignatures
\pysiglinewithargsret{\sphinxbfcode{\sphinxupquote{pressure}}}{\sphinxparam{\DUrole{n}{logT}}\sphinxparamcomma \sphinxparam{\DUrole{n}{logRho}}}{}
\pysigstopsignatures
\sphinxAtStartPar
Find the pressure at the given temperature and density.
\begin{quote}\begin{description}
\sphinxlineitem{Parameters}\begin{description}
\sphinxlineitem{\sphinxstylestrong{logT}}{[}float{]}
\sphinxAtStartPar
Log10 of the temperature in K

\sphinxlineitem{\sphinxstylestrong{logRho}}{[}float{]}
\sphinxAtStartPar
Log10 of the density in g/cm\textasciicircum{}3

\end{description}

\end{description}\end{quote}

\end{fulllineitems}

\index{rhoRange (CoolDwarf.EOS.ChabrierDebras2021.EOS.CH21EOS property)@\spxentry{rhoRange}\spxextra{CoolDwarf.EOS.ChabrierDebras2021.EOS.CH21EOS property}}

\begin{fulllineitems}
\phantomsection\label{\detokenize{CoolDwarf.EOS.ChabrierDebras2021:CoolDwarf.EOS.ChabrierDebras2021.EOS.CH21EOS.rhoRange}}
\pysigstartsignatures
\pysigline{\sphinxbfcode{\sphinxupquote{property\DUrole{w}{ }}}\sphinxbfcode{\sphinxupquote{rhoRange}}}
\pysigstopsignatures
\sphinxAtStartPar
Tuple containing the minimum and maximum density (log10(ρ)) in the EOS table
\begin{quote}\begin{description}
\sphinxlineitem{Returns}\begin{description}
\sphinxlineitem{tuple}
\sphinxAtStartPar
Tuple containing the minimum and maximum density (log10(ρ)) in the EOS table

\end{description}

\end{description}\end{quote}

\end{fulllineitems}


\end{fulllineitems}



\subparagraph{Module contents}
\label{\detokenize{CoolDwarf.EOS.ChabrierDebras2021:module-CoolDwarf.EOS.ChabrierDebras2021}}\label{\detokenize{CoolDwarf.EOS.ChabrierDebras2021:module-contents}}\index{module@\spxentry{module}!CoolDwarf.EOS.ChabrierDebras2021@\spxentry{CoolDwarf.EOS.ChabrierDebras2021}}\index{CoolDwarf.EOS.ChabrierDebras2021@\spxentry{CoolDwarf.EOS.ChabrierDebras2021}!module@\spxentry{module}}
\sphinxstepscope


\subparagraph{CoolDwarf.EOS.invert package}
\label{\detokenize{CoolDwarf.EOS.invert:cooldwarf-eos-invert-package}}\label{\detokenize{CoolDwarf.EOS.invert::doc}}

\subparagraph{Submodules}
\label{\detokenize{CoolDwarf.EOS.invert:submodules}}

\subparagraph{CoolDwarf.EOS.invert.EOSInverter module}
\label{\detokenize{CoolDwarf.EOS.invert:module-CoolDwarf.EOS.invert.EOSInverter}}\label{\detokenize{CoolDwarf.EOS.invert:cooldwarf-eos-invert-eosinverter-module}}\index{module@\spxentry{module}!CoolDwarf.EOS.invert.EOSInverter@\spxentry{CoolDwarf.EOS.invert.EOSInverter}}\index{CoolDwarf.EOS.invert.EOSInverter@\spxentry{CoolDwarf.EOS.invert.EOSInverter}!module@\spxentry{module}}
\sphinxAtStartPar
EOSInverter.py \textendash{} Inverter class for EOS tables

\sphinxAtStartPar
This module contains the Inverter class for EOS tables. The class is designed to be used with the CoolDwarf Stellar Structure code, and provides the necessary functions to invert the EOS tables.
Because the inversion problem is non\sphinxhyphen{}linear, the Inverter class uses the scipy.optimize.minimize function to find the solution.

\sphinxAtStartPar
Further, because EOSs may not be truley invertible, the Inverter class uses a loss function to find the closest solution to the target energy.
over a limited range of temperatures and densities. This is intended to be a range centered around the initial guess for the inversion and
limited in size by some expected maximum deviation from the initial guess.


\subparagraph{Dependencies}
\label{\detokenize{CoolDwarf.EOS.invert:dependencies}}\begin{itemize}
\item {} 
\sphinxAtStartPar
cupy

\item {} 
\sphinxAtStartPar
torch

\item {} 
\sphinxAtStartPar
CoolDwarf.err

\end{itemize}


\subparagraph{Example usage}
\label{\detokenize{CoolDwarf.EOS.invert:example-usage}}
\begin{sphinxVerbatim}[commandchars=\\\{\}]
\PYG{g+gp}{\PYGZgt{}\PYGZgt{}\PYGZgt{} }\PYG{k+kn}{from} \PYG{n+nn}{CoolDwarf}\PYG{n+nn}{.}\PYG{n+nn}{EOS}\PYG{n+nn}{.}\PYG{n+nn}{invert}\PYG{n+nn}{.}\PYG{n+nn}{EOSInverter} \PYG{k+kn}{import} \PYG{n}{Inverter}
\PYG{g+gp}{\PYGZgt{}\PYGZgt{}\PYGZgt{} }\PYG{k+kn}{from} \PYG{n+nn}{CoolDwarf}\PYG{n+nn}{.}\PYG{n+nn}{EOS}\PYG{n+nn}{.}\PYG{n+nn}{ChabrierDebras2021}\PYG{n+nn}{.}\PYG{n+nn}{EOS} \PYG{k+kn}{import} \PYG{n}{CH21EOS}
\PYG{g+gp}{\PYGZgt{}\PYGZgt{}\PYGZgt{} }\PYG{n}{eos} \PYG{o}{=} \PYG{n}{CH21EOS}\PYG{p}{(}\PYG{l+s+s2}{\PYGZdq{}}\PYG{l+s+s2}{path/to/eos/table}\PYG{l+s+s2}{\PYGZdq{}}\PYG{p}{)}
\PYG{g+gp}{\PYGZgt{}\PYGZgt{}\PYGZgt{} }\PYG{n}{inverter} \PYG{o}{=} \PYG{n}{Inverter}\PYG{p}{(}\PYG{n}{eos}\PYG{p}{,} \PYG{n}{TRange}\PYG{p}{,} \PYG{n}{RhoRange}\PYG{p}{)}
\PYG{g+gp}{\PYGZgt{}\PYGZgt{}\PYGZgt{} }\PYG{n}{logTInit}\PYG{p}{,} \PYG{n}{logRhoInit} \PYG{o}{=} \PYG{l+m+mf}{7.0}\PYG{p}{,} \PYG{o}{\PYGZhy{}}\PYG{l+m+mf}{2.0}
\PYG{g+gp}{\PYGZgt{}\PYGZgt{}\PYGZgt{} }\PYG{n}{newTRange} \PYG{o}{=} \PYG{p}{(}\PYG{l+m+mf}{6.0}\PYG{p}{,} \PYG{l+m+mf}{8.0}\PYG{p}{)}
\PYG{g+gp}{\PYGZgt{}\PYGZgt{}\PYGZgt{} }\PYG{n}{newRhoRange} \PYG{o}{=} \PYG{p}{(}\PYG{o}{\PYGZhy{}}\PYG{l+m+mf}{3.0}\PYG{p}{,} \PYG{l+m+mf}{0.0}\PYG{p}{)}
\PYG{g+gp}{\PYGZgt{}\PYGZgt{}\PYGZgt{} }\PYG{n}{energy} \PYG{o}{=} \PYG{l+m+mf}{1e15}
\PYG{g+gp}{\PYGZgt{}\PYGZgt{}\PYGZgt{} }\PYG{n}{newBounds} \PYG{o}{=} \PYG{p}{(}\PYG{n}{newTRange}\PYG{p}{,} \PYG{n}{newRhoRange}\PYG{p}{)}
\PYG{g+gp}{\PYGZgt{}\PYGZgt{}\PYGZgt{} }\PYG{n}{inverter}\PYG{o}{.}\PYG{n}{set\PYGZus{}bounds}\PYG{p}{(}\PYG{n}{newBounds}\PYG{p}{)}
\PYG{g+gp}{\PYGZgt{}\PYGZgt{}\PYGZgt{} }\PYG{n}{logT}\PYG{p}{,} \PYG{n}{logRho} \PYG{o}{=} \PYG{n}{inverter}\PYG{o}{.}\PYG{n}{temperature\PYGZus{}density}\PYG{p}{(}\PYG{n}{energy}\PYG{p}{,} \PYG{n}{logTInit}\PYG{p}{,} \PYG{n}{logRhoInit}\PYG{p}{)}
\end{sphinxVerbatim}
\index{Inverter (class in CoolDwarf.EOS.invert.EOSInverter)@\spxentry{Inverter}\spxextra{class in CoolDwarf.EOS.invert.EOSInverter}}

\begin{fulllineitems}
\phantomsection\label{\detokenize{CoolDwarf.EOS.invert:CoolDwarf.EOS.invert.EOSInverter.Inverter}}
\pysigstartsignatures
\pysiglinewithargsret{\sphinxbfcode{\sphinxupquote{class\DUrole{w}{ }}}\sphinxcode{\sphinxupquote{CoolDwarf.EOS.invert.EOSInverter.}}\sphinxbfcode{\sphinxupquote{Inverter}}}{\sphinxparam{\DUrole{n}{EOS}}}{}
\pysigstopsignatures
\sphinxAtStartPar
Bases: \sphinxcode{\sphinxupquote{object}}

\sphinxAtStartPar
Inverter \textendash{} Inverter class for EOS tables

\sphinxAtStartPar
This class is designed to be used with the CoolDwarf Stellar Structure code, and provides the necessary functions
to invert the EOS tables. The Inverter class uses PyTorch optimizers to find the solution to the non\sphinxhyphen{}linear inversion problem.
Because EOSs may not be truly invertible, the Inverter class uses a loss function to find the closest solution to the target energy
over a limited range of temperatures and densities. This is intended to be a range centered around the initial guess for the inversion
and limited in size by some expected maximum deviation from the initial guess.
\begin{quote}\begin{description}
\sphinxlineitem{Parameters}\begin{description}
\sphinxlineitem{\sphinxstylestrong{EOS}}{[}EOS{]}
\sphinxAtStartPar
EOS object to invert

\sphinxlineitem{\sphinxstylestrong{TRange}}{[}tuple{]}
\sphinxAtStartPar
Tuple containing the minimum and maximum temperature (log10(T)) in the EOS table

\sphinxlineitem{\sphinxstylestrong{RhoRange}}{[}tuple{]}
\sphinxAtStartPar
Tuple containing the minimum and maximum density (log10(ρ)) in the EOS table

\end{description}

\sphinxlineitem{Attributes}\begin{description}
\sphinxlineitem{\sphinxstylestrong{EOS}}{[}EOS{]}
\sphinxAtStartPar
EOS object to invert

\sphinxlineitem{\sphinxstylestrong{\_TRange}}{[}tuple{]}
\sphinxAtStartPar
Tuple containing the minimum and maximum temperature (log10(T)) in the EOS table

\sphinxlineitem{\sphinxstylestrong{\_RhoRange}}{[}tuple{]}
\sphinxAtStartPar
Tuple containing the minimum and maximum density (log10(ρ)) in the EOS table

\sphinxlineitem{\sphinxstylestrong{\_bounds}}{[}tuple{]}
\sphinxAtStartPar
Tuple containing the TRange and RhoRange

\end{description}

\end{description}\end{quote}
\subsubsection*{Methods}


\begin{savenotes}\sphinxattablestart
\sphinxthistablewithglobalstyle
\centering
\begin{tabulary}{\linewidth}[t]{TT}
\sphinxtoprule
\sphinxtableatstartofbodyhook
\sphinxAtStartPar
\sphinxstylestrong{temperature\_density(energy, logTInit, logRhoInit)}
&
\sphinxAtStartPar
Inverts the EOS to find the temperature and density that gives the target energy
\\
\sphinxhline
\sphinxAtStartPar
\sphinxstylestrong{set\_bounds(newBounds)}
&
\sphinxAtStartPar
Sets the bounds for the inversion
\\
\sphinxbottomrule
\end{tabulary}
\sphinxtableafterendhook\par
\sphinxattableend\end{savenotes}
\index{set\_bounds() (CoolDwarf.EOS.invert.EOSInverter.Inverter method)@\spxentry{set\_bounds()}\spxextra{CoolDwarf.EOS.invert.EOSInverter.Inverter method}}

\begin{fulllineitems}
\phantomsection\label{\detokenize{CoolDwarf.EOS.invert:CoolDwarf.EOS.invert.EOSInverter.Inverter.set_bounds}}
\pysigstartsignatures
\pysiglinewithargsret{\sphinxbfcode{\sphinxupquote{set\_bounds}}}{\sphinxparam{\DUrole{n}{tRange}}\sphinxparamcomma \sphinxparam{\DUrole{n}{rRange}}}{}
\pysigstopsignatures
\end{fulllineitems}

\index{temperature\_density() (CoolDwarf.EOS.invert.EOSInverter.Inverter method)@\spxentry{temperature\_density()}\spxextra{CoolDwarf.EOS.invert.EOSInverter.Inverter method}}

\begin{fulllineitems}
\phantomsection\label{\detokenize{CoolDwarf.EOS.invert:CoolDwarf.EOS.invert.EOSInverter.Inverter.temperature_density}}
\pysigstartsignatures
\pysiglinewithargsret{\sphinxbfcode{\sphinxupquote{temperature\_density}}}{\sphinxparam{\DUrole{n}{energy}\DUrole{p}{:}\DUrole{w}{ }\DUrole{n}{Tensor}}\sphinxparamcomma \sphinxparam{\DUrole{n}{logTInit}\DUrole{p}{:}\DUrole{w}{ }\DUrole{n}{Tensor}}\sphinxparamcomma \sphinxparam{\DUrole{n}{logRhoInit}\DUrole{p}{:}\DUrole{w}{ }\DUrole{n}{Tensor}}\sphinxparamcomma \sphinxparam{\DUrole{n}{lr}\DUrole{p}{:}\DUrole{w}{ }\DUrole{n}{float}\DUrole{w}{ }\DUrole{o}{=}\DUrole{w}{ }\DUrole{default_value}{0.01}}\sphinxparamcomma \sphinxparam{\DUrole{n}{num\_epochs}\DUrole{p}{:}\DUrole{w}{ }\DUrole{n}{int}\DUrole{w}{ }\DUrole{o}{=}\DUrole{w}{ }\DUrole{default_value}{1000}}}{{ $\rightarrow$ Tensor}}
\pysigstopsignatures
\end{fulllineitems}


\end{fulllineitems}

\index{cupy\_to\_torch() (in module CoolDwarf.EOS.invert.EOSInverter)@\spxentry{cupy\_to\_torch()}\spxextra{in module CoolDwarf.EOS.invert.EOSInverter}}

\begin{fulllineitems}
\phantomsection\label{\detokenize{CoolDwarf.EOS.invert:CoolDwarf.EOS.invert.EOSInverter.cupy_to_torch}}
\pysigstartsignatures
\pysiglinewithargsret{\sphinxcode{\sphinxupquote{CoolDwarf.EOS.invert.EOSInverter.}}\sphinxbfcode{\sphinxupquote{cupy\_to\_torch}}}{\sphinxparam{\DUrole{n}{cupy\_array}}}{}
\pysigstopsignatures
\end{fulllineitems}



\subparagraph{Module contents}
\label{\detokenize{CoolDwarf.EOS.invert:module-CoolDwarf.EOS.invert}}\label{\detokenize{CoolDwarf.EOS.invert:module-contents}}\index{module@\spxentry{module}!CoolDwarf.EOS.invert@\spxentry{CoolDwarf.EOS.invert}}\index{CoolDwarf.EOS.invert@\spxentry{CoolDwarf.EOS.invert}!module@\spxentry{module}}

\paragraph{Submodules}
\label{\detokenize{CoolDwarf.EOS:submodules}}

\paragraph{CoolDwarf.EOS.EOS module}
\label{\detokenize{CoolDwarf.EOS:module-CoolDwarf.EOS.EOS}}\label{\detokenize{CoolDwarf.EOS:cooldwarf-eos-eos-module}}\index{module@\spxentry{module}!CoolDwarf.EOS.EOS@\spxentry{CoolDwarf.EOS.EOS}}\index{CoolDwarf.EOS.EOS@\spxentry{CoolDwarf.EOS.EOS}!module@\spxentry{module}}
\sphinxAtStartPar
EOS.py \textendash{} General EOS retreival function for CoolDwarf

\sphinxAtStartPar
This module contains the get\_eos function, which is used to retrieve the appropriate EOS object based on the format
of the EOS table.


\subparagraph{Dependencies}
\label{\detokenize{CoolDwarf.EOS:dependencies}}\begin{itemize}
\item {} 
\sphinxAtStartPar
CoolDwarf.EOS.ChabrierDebras2021.EOS

\item {} 
\sphinxAtStartPar
CoolDwarf.err

\end{itemize}


\subparagraph{Example usage}
\label{\detokenize{CoolDwarf.EOS:example-usage}}
\begin{sphinxVerbatim}[commandchars=\\\{\}]
\PYG{g+gp}{\PYGZgt{}\PYGZgt{}\PYGZgt{} }\PYG{k+kn}{from} \PYG{n+nn}{CoolDwarf}\PYG{n+nn}{.}\PYG{n+nn}{EOS}\PYG{n+nn}{.}\PYG{n+nn}{EOS} \PYG{k+kn}{import} \PYG{n}{get\PYGZus{}eos}
\PYG{g+gp}{\PYGZgt{}\PYGZgt{}\PYGZgt{} }\PYG{n}{eos} \PYG{o}{=} \PYG{n}{get\PYGZus{}eos}\PYG{p}{(}\PYG{l+s+s2}{\PYGZdq{}}\PYG{l+s+s2}{path/to/eos/table}\PYG{l+s+s2}{\PYGZdq{}}\PYG{p}{,} \PYG{l+s+s2}{\PYGZdq{}}\PYG{l+s+s2}{CD21}\PYG{l+s+s2}{\PYGZdq{}}\PYG{p}{)}
\end{sphinxVerbatim}
\index{get\_eos() (in module CoolDwarf.EOS.EOS)@\spxentry{get\_eos()}\spxextra{in module CoolDwarf.EOS.EOS}}

\begin{fulllineitems}
\phantomsection\label{\detokenize{CoolDwarf.EOS:CoolDwarf.EOS.EOS.get_eos}}
\pysigstartsignatures
\pysiglinewithargsret{\sphinxcode{\sphinxupquote{CoolDwarf.EOS.EOS.}}\sphinxbfcode{\sphinxupquote{get\_eos}}}{\sphinxparam{\DUrole{n}{path}\DUrole{p}{:}\DUrole{w}{ }\DUrole{n}{str}}\sphinxparamcomma \sphinxparam{\DUrole{n}{format}\DUrole{p}{:}\DUrole{w}{ }\DUrole{n}{str}}}{}
\pysigstopsignatures
\sphinxAtStartPar
This function is used to retrieve the appropriate EOS object based on the format of the EOS table.
Available formats are:
\sphinxhyphen{} CD21: Chabrier Debras 2021 EOS tables
\begin{quote}\begin{description}
\sphinxlineitem{Parameters}\begin{description}
\sphinxlineitem{\sphinxstylestrong{path}}{[}str{]}
\sphinxAtStartPar
Path to the EOS table

\sphinxlineitem{\sphinxstylestrong{format}}{[}str{]}
\sphinxAtStartPar
Format of the EOS table. Available formats include: CD21 for Chabrier Debras 2021 EOS tables.

\end{description}

\sphinxlineitem{Returns}\begin{description}
\sphinxlineitem{\sphinxstylestrong{EOS}}{[}EOS{]}
\sphinxAtStartPar
EOS object for the given EOS table

\end{description}

\sphinxlineitem{Raises}\begin{description}
\sphinxlineitem{EOSFormatError}
\sphinxAtStartPar
If the format is not recognized

\end{description}

\end{description}\end{quote}

\end{fulllineitems}



\paragraph{Module contents}
\label{\detokenize{CoolDwarf.EOS:module-CoolDwarf.EOS}}\label{\detokenize{CoolDwarf.EOS:module-contents}}\index{module@\spxentry{module}!CoolDwarf.EOS@\spxentry{CoolDwarf.EOS}}\index{CoolDwarf.EOS@\spxentry{CoolDwarf.EOS}!module@\spxentry{module}}
\sphinxstepscope


\subsubsection{CoolDwarf.err package}
\label{\detokenize{CoolDwarf.err:cooldwarf-err-package}}\label{\detokenize{CoolDwarf.err::doc}}

\paragraph{Submodules}
\label{\detokenize{CoolDwarf.err:submodules}}

\paragraph{CoolDwarf.err.energy module}
\label{\detokenize{CoolDwarf.err:module-CoolDwarf.err.energy}}\label{\detokenize{CoolDwarf.err:cooldwarf-err-energy-module}}\index{module@\spxentry{module}!CoolDwarf.err.energy@\spxentry{CoolDwarf.err.energy}}\index{CoolDwarf.err.energy@\spxentry{CoolDwarf.err.energy}!module@\spxentry{module}}\index{EnergyConservationError@\spxentry{EnergyConservationError}}

\begin{fulllineitems}
\phantomsection\label{\detokenize{CoolDwarf.err:CoolDwarf.err.energy.EnergyConservationError}}
\pysigstartsignatures
\pysiglinewithargsret{\sphinxbfcode{\sphinxupquote{exception\DUrole{w}{ }}}\sphinxcode{\sphinxupquote{CoolDwarf.err.energy.}}\sphinxbfcode{\sphinxupquote{EnergyConservationError}}}{\sphinxparam{\DUrole{n}{msg}}}{}
\pysigstopsignatures
\sphinxAtStartPar
Bases: \sphinxcode{\sphinxupquote{Exception}}

\sphinxAtStartPar
An exception class for energy conservation error. This exception is raised when the energy conservation is not
satisfied during integration of the model.

\end{fulllineitems}

\index{NonConvergenceError@\spxentry{NonConvergenceError}}

\begin{fulllineitems}
\phantomsection\label{\detokenize{CoolDwarf.err:CoolDwarf.err.energy.NonConvergenceError}}
\pysigstartsignatures
\pysiglinewithargsret{\sphinxbfcode{\sphinxupquote{exception\DUrole{w}{ }}}\sphinxcode{\sphinxupquote{CoolDwarf.err.energy.}}\sphinxbfcode{\sphinxupquote{NonConvergenceError}}}{\sphinxparam{\DUrole{n}{msg}}}{}
\pysigstopsignatures
\sphinxAtStartPar
Bases: \sphinxcode{\sphinxupquote{Exception}}

\sphinxAtStartPar
An exception class for non\sphinxhyphen{}convergence error. This exception is raised when the solver does not converge.

\end{fulllineitems}



\paragraph{CoolDwarf.err.eos module}
\label{\detokenize{CoolDwarf.err:module-CoolDwarf.err.eos}}\label{\detokenize{CoolDwarf.err:cooldwarf-err-eos-module}}\index{module@\spxentry{module}!CoolDwarf.err.eos@\spxentry{CoolDwarf.err.eos}}\index{CoolDwarf.err.eos@\spxentry{CoolDwarf.err.eos}!module@\spxentry{module}}\index{EOSBoundsError@\spxentry{EOSBoundsError}}

\begin{fulllineitems}
\phantomsection\label{\detokenize{CoolDwarf.err:CoolDwarf.err.eos.EOSBoundsError}}
\pysigstartsignatures
\pysiglinewithargsret{\sphinxbfcode{\sphinxupquote{exception\DUrole{w}{ }}}\sphinxcode{\sphinxupquote{CoolDwarf.err.eos.}}\sphinxbfcode{\sphinxupquote{EOSBoundsError}}}{\sphinxparam{\DUrole{n}{msg}}}{}
\pysigstopsignatures
\sphinxAtStartPar
Bases: \sphinxcode{\sphinxupquote{Exception}}

\sphinxAtStartPar
An exception class for EOS bounds error. This exception is raised when the bounds for the inversion are not valid.

\end{fulllineitems}

\index{EOSFormatError@\spxentry{EOSFormatError}}

\begin{fulllineitems}
\phantomsection\label{\detokenize{CoolDwarf.err:CoolDwarf.err.eos.EOSFormatError}}
\pysigstartsignatures
\pysiglinewithargsret{\sphinxbfcode{\sphinxupquote{exception\DUrole{w}{ }}}\sphinxcode{\sphinxupquote{CoolDwarf.err.eos.}}\sphinxbfcode{\sphinxupquote{EOSFormatError}}}{\sphinxparam{\DUrole{n}{msg}}}{}
\pysigstopsignatures
\sphinxAtStartPar
Bases: \sphinxcode{\sphinxupquote{Exception}}

\sphinxAtStartPar
An exception class for EOS format error. This exception is raised when the EOS format is not recognized.

\end{fulllineitems}

\index{EOSInverterError@\spxentry{EOSInverterError}}

\begin{fulllineitems}
\phantomsection\label{\detokenize{CoolDwarf.err:CoolDwarf.err.eos.EOSInverterError}}
\pysigstartsignatures
\pysiglinewithargsret{\sphinxbfcode{\sphinxupquote{exception\DUrole{w}{ }}}\sphinxcode{\sphinxupquote{CoolDwarf.err.eos.}}\sphinxbfcode{\sphinxupquote{EOSInverterError}}}{\sphinxparam{\DUrole{n}{msg}}}{}
\pysigstopsignatures
\sphinxAtStartPar
Bases: \sphinxcode{\sphinxupquote{Exception}}

\sphinxAtStartPar
An exception class for EOS inverter error. This exception is raised when an error occurs during the inversion of the EOS.

\end{fulllineitems}



\paragraph{Module contents}
\label{\detokenize{CoolDwarf.err:module-CoolDwarf.err}}\label{\detokenize{CoolDwarf.err:module-contents}}\index{module@\spxentry{module}!CoolDwarf.err@\spxentry{CoolDwarf.err}}\index{CoolDwarf.err@\spxentry{CoolDwarf.err}!module@\spxentry{module}}
\sphinxstepscope


\subsubsection{CoolDwarf.ext package}
\label{\detokenize{CoolDwarf.ext:cooldwarf-ext-package}}\label{\detokenize{CoolDwarf.ext::doc}}

\paragraph{Module contents}
\label{\detokenize{CoolDwarf.ext:module-CoolDwarf.ext}}\label{\detokenize{CoolDwarf.ext:module-contents}}\index{module@\spxentry{module}!CoolDwarf.ext@\spxentry{CoolDwarf.ext}}\index{CoolDwarf.ext@\spxentry{CoolDwarf.ext}!module@\spxentry{module}}
\sphinxstepscope


\subsubsection{CoolDwarf.model package}
\label{\detokenize{CoolDwarf.model:cooldwarf-model-package}}\label{\detokenize{CoolDwarf.model::doc}}

\paragraph{Subpackages}
\label{\detokenize{CoolDwarf.model:subpackages}}
\sphinxstepscope


\subparagraph{CoolDwarf.model.dsep package}
\label{\detokenize{CoolDwarf.model.dsep:cooldwarf-model-dsep-package}}\label{\detokenize{CoolDwarf.model.dsep::doc}}

\subparagraph{Submodules}
\label{\detokenize{CoolDwarf.model.dsep:submodules}}

\subparagraph{CoolDwarf.model.dsep.dsep module}
\label{\detokenize{CoolDwarf.model.dsep:module-CoolDwarf.model.dsep.dsep}}\label{\detokenize{CoolDwarf.model.dsep:cooldwarf-model-dsep-dsep-module}}\index{module@\spxentry{module}!CoolDwarf.model.dsep.dsep@\spxentry{CoolDwarf.model.dsep.dsep}}\index{CoolDwarf.model.dsep.dsep@\spxentry{CoolDwarf.model.dsep.dsep}!module@\spxentry{module}}\index{parse\_dsep\_MOD\_file() (in module CoolDwarf.model.dsep.dsep)@\spxentry{parse\_dsep\_MOD\_file()}\spxextra{in module CoolDwarf.model.dsep.dsep}}

\begin{fulllineitems}
\phantomsection\label{\detokenize{CoolDwarf.model.dsep:CoolDwarf.model.dsep.dsep.parse_dsep_MOD_file}}
\pysigstartsignatures
\pysiglinewithargsret{\sphinxcode{\sphinxupquote{CoolDwarf.model.dsep.dsep.}}\sphinxbfcode{\sphinxupquote{parse\_dsep\_MOD\_file}}}{}{}
\pysigstopsignatures
\end{fulllineitems}



\subparagraph{Module contents}
\label{\detokenize{CoolDwarf.model.dsep:module-CoolDwarf.model.dsep}}\label{\detokenize{CoolDwarf.model.dsep:module-contents}}\index{module@\spxentry{module}!CoolDwarf.model.dsep@\spxentry{CoolDwarf.model.dsep}}\index{CoolDwarf.model.dsep@\spxentry{CoolDwarf.model.dsep}!module@\spxentry{module}}
\sphinxstepscope


\subparagraph{CoolDwarf.model.mesa package}
\label{\detokenize{CoolDwarf.model.mesa:cooldwarf-model-mesa-package}}\label{\detokenize{CoolDwarf.model.mesa::doc}}

\subparagraph{Submodules}
\label{\detokenize{CoolDwarf.model.mesa:submodules}}

\subparagraph{CoolDwarf.model.mesa.mesa module}
\label{\detokenize{CoolDwarf.model.mesa:module-CoolDwarf.model.mesa.mesa}}\label{\detokenize{CoolDwarf.model.mesa:cooldwarf-model-mesa-mesa-module}}\index{module@\spxentry{module}!CoolDwarf.model.mesa.mesa@\spxentry{CoolDwarf.model.mesa.mesa}}\index{CoolDwarf.model.mesa.mesa@\spxentry{CoolDwarf.model.mesa.mesa}!module@\spxentry{module}}
\sphinxAtStartPar
mesa.py \textendash{} MESA MOD file parser

\sphinxAtStartPar
This module contains a function to parse MESA MOD files. The function reads the file and extracts the metadata and data sections.
Due to the format of the MESA MOD files, the metadata section is read line by line, while the data section is read as a fixed\sphinxhyphen{}width file.
The data is then stored in a pandas DataFrame. Finally, because MESA uses D instead of E for scientific notation, the function replaces D with E in the data section.


\subparagraph{Dependencies}
\label{\detokenize{CoolDwarf.model.mesa:dependencies}}\begin{itemize}
\item {} 
\sphinxAtStartPar
pandas

\end{itemize}


\subparagraph{Example usage}
\label{\detokenize{CoolDwarf.model.mesa:example-usage}}
\begin{sphinxVerbatim}[commandchars=\\\{\}]
\PYG{g+gp}{\PYGZgt{}\PYGZgt{}\PYGZgt{} }\PYG{k+kn}{from} \PYG{n+nn}{CoolDwarf}\PYG{n+nn}{.}\PYG{n+nn}{model}\PYG{n+nn}{.}\PYG{n+nn}{mesa}\PYG{n+nn}{.}\PYG{n+nn}{mesa} \PYG{k+kn}{import} \PYG{n}{parse\PYGZus{}mesa\PYGZus{}MOD\PYGZus{}file}
\PYG{g+gp}{\PYGZgt{}\PYGZgt{}\PYGZgt{} }\PYG{n}{df} \PYG{o}{=} \PYG{n}{parse\PYGZus{}mesa\PYGZus{}MOD\PYGZus{}file}\PYG{p}{(}\PYG{l+s+s2}{\PYGZdq{}}\PYG{l+s+s2}{path/to/mod/file}\PYG{l+s+s2}{\PYGZdq{}}\PYG{p}{)}
\PYG{g+gp}{\PYGZgt{}\PYGZgt{}\PYGZgt{} }\PYG{n+nb}{print}\PYG{p}{(}\PYG{n}{df}\PYG{p}{)}
\end{sphinxVerbatim}
\index{parse\_mesa\_MOD\_file() (in module CoolDwarf.model.mesa.mesa)@\spxentry{parse\_mesa\_MOD\_file()}\spxextra{in module CoolDwarf.model.mesa.mesa}}

\begin{fulllineitems}
\phantomsection\label{\detokenize{CoolDwarf.model.mesa:CoolDwarf.model.mesa.mesa.parse_mesa_MOD_file}}
\pysigstartsignatures
\pysiglinewithargsret{\sphinxcode{\sphinxupquote{CoolDwarf.model.mesa.mesa.}}\sphinxbfcode{\sphinxupquote{parse\_mesa\_MOD\_file}}}{\sphinxparam{\DUrole{n}{filepath}\DUrole{p}{:}\DUrole{w}{ }\DUrole{n}{str}}}{{ $\rightarrow$ DataFrame}}
\pysigstopsignatures
\sphinxAtStartPar
This function reads a MESA MOD file and extracts the metadata and data sections. The data is then stored in a pandas DataFrame.
\begin{quote}\begin{description}
\sphinxlineitem{Parameters}\begin{description}
\sphinxlineitem{\sphinxstylestrong{filepath}}{[}str{]}
\sphinxAtStartPar
Path to the MESA MOD file

\end{description}

\sphinxlineitem{Returns}\begin{description}
\sphinxlineitem{\sphinxstylestrong{df}}{[}pd.DataFrame{]}
\sphinxAtStartPar
DataFrame containing the data from the MESA MOD file

\end{description}

\end{description}\end{quote}

\end{fulllineitems}



\subparagraph{Module contents}
\label{\detokenize{CoolDwarf.model.mesa:module-CoolDwarf.model.mesa}}\label{\detokenize{CoolDwarf.model.mesa:module-contents}}\index{module@\spxentry{module}!CoolDwarf.model.mesa@\spxentry{CoolDwarf.model.mesa}}\index{CoolDwarf.model.mesa@\spxentry{CoolDwarf.model.mesa}!module@\spxentry{module}}

\paragraph{Submodules}
\label{\detokenize{CoolDwarf.model:submodules}}

\paragraph{CoolDwarf.model.model module}
\label{\detokenize{CoolDwarf.model:module-CoolDwarf.model.model}}\label{\detokenize{CoolDwarf.model:cooldwarf-model-model-module}}\index{module@\spxentry{module}!CoolDwarf.model.model@\spxentry{CoolDwarf.model.model}}\index{CoolDwarf.model.model@\spxentry{CoolDwarf.model.model}!module@\spxentry{module}}
\sphinxAtStartPar
model.py \textendash{} General model retreival function for CoolDwarf

\sphinxAtStartPar
This module contains the get\_model function, which is used to retrieve the appropriate model object based on the format
of the model file.


\subparagraph{Dependencies}
\label{\detokenize{CoolDwarf.model:dependencies}}\begin{itemize}
\item {} 
\sphinxAtStartPar
pandas

\item {} 
\sphinxAtStartPar
CoolDwarf.model.mesa

\item {} 
\sphinxAtStartPar
CoolDwarf.model.dsep

\end{itemize}


\subparagraph{Example usage}
\label{\detokenize{CoolDwarf.model:example-usage}}
\begin{sphinxVerbatim}[commandchars=\\\{\}]
\PYG{g+gp}{\PYGZgt{}\PYGZgt{}\PYGZgt{} }\PYG{k+kn}{from} \PYG{n+nn}{CoolDwarf}\PYG{n+nn}{.}\PYG{n+nn}{model}\PYG{n+nn}{.}\PYG{n+nn}{model} \PYG{k+kn}{import} \PYG{n}{get\PYGZus{}model}
\PYG{g+gp}{\PYGZgt{}\PYGZgt{}\PYGZgt{} }\PYG{n}{model} \PYG{o}{=} \PYG{n}{get\PYGZus{}model}\PYG{p}{(}\PYG{l+s+s2}{\PYGZdq{}}\PYG{l+s+s2}{path/to/model/file}\PYG{l+s+s2}{\PYGZdq{}}\PYG{p}{,} \PYG{l+s+s2}{\PYGZdq{}}\PYG{l+s+s2}{mesa}\PYG{l+s+s2}{\PYGZdq{}}\PYG{p}{)}
\PYG{g+gp}{\PYGZgt{}\PYGZgt{}\PYGZgt{} }\PYG{n+nb}{print}\PYG{p}{(}\PYG{n}{model}\PYG{p}{)}
\end{sphinxVerbatim}
\index{get\_model() (in module CoolDwarf.model.model)@\spxentry{get\_model()}\spxextra{in module CoolDwarf.model.model}}

\begin{fulllineitems}
\phantomsection\label{\detokenize{CoolDwarf.model:CoolDwarf.model.model.get_model}}
\pysigstartsignatures
\pysiglinewithargsret{\sphinxcode{\sphinxupquote{CoolDwarf.model.model.}}\sphinxbfcode{\sphinxupquote{get\_model}}}{\sphinxparam{\DUrole{n}{path}\DUrole{p}{:}\DUrole{w}{ }\DUrole{n}{str}}\sphinxparamcomma \sphinxparam{\DUrole{n}{format}\DUrole{p}{:}\DUrole{w}{ }\DUrole{n}{str}}}{{ $\rightarrow$ DataFrame}}
\pysigstopsignatures
\sphinxAtStartPar
This function is used to retrieve the appropriate model object based on the format of the model file.
Available formats are:
\sphinxhyphen{} mesa: MESA MOD files
\sphinxhyphen{} dsep: DSEP MOD files
\begin{quote}\begin{description}
\sphinxlineitem{Parameters}\begin{description}
\sphinxlineitem{\sphinxstylestrong{path}}{[}str{]}
\sphinxAtStartPar
Path to the model file

\sphinxlineitem{\sphinxstylestrong{format}}{[}str{]}
\sphinxAtStartPar
Format of the model file. Available formats include: mesa for MESA MOD files, dsep for DSEP MOD files.

\end{description}

\sphinxlineitem{Returns}\begin{description}
\sphinxlineitem{\sphinxstylestrong{model}}{[}pd.DataFrame{]}
\sphinxAtStartPar
DataFrame containing the model data

\end{description}

\sphinxlineitem{Raises}\begin{description}
\sphinxlineitem{SSEModelError}
\sphinxAtStartPar
If the format is not recognized

\end{description}

\end{description}\end{quote}

\end{fulllineitems}



\paragraph{Module contents}
\label{\detokenize{CoolDwarf.model:module-CoolDwarf.model}}\label{\detokenize{CoolDwarf.model:module-contents}}\index{module@\spxentry{module}!CoolDwarf.model@\spxentry{CoolDwarf.model}}\index{CoolDwarf.model@\spxentry{CoolDwarf.model}!module@\spxentry{module}}
\sphinxstepscope


\subsubsection{CoolDwarf.opac package}
\label{\detokenize{CoolDwarf.opac:cooldwarf-opac-package}}\label{\detokenize{CoolDwarf.opac::doc}}

\paragraph{Subpackages}
\label{\detokenize{CoolDwarf.opac:subpackages}}
\sphinxstepscope


\subparagraph{CoolDwarf.opac.aesopus package}
\label{\detokenize{CoolDwarf.opac.aesopus:cooldwarf-opac-aesopus-package}}\label{\detokenize{CoolDwarf.opac.aesopus::doc}}

\subparagraph{Submodules}
\label{\detokenize{CoolDwarf.opac.aesopus:submodules}}

\subparagraph{CoolDwarf.opac.aesopus.load module}
\label{\detokenize{CoolDwarf.opac.aesopus:module-CoolDwarf.opac.aesopus.load}}\label{\detokenize{CoolDwarf.opac.aesopus:cooldwarf-opac-aesopus-load-module}}\index{module@\spxentry{module}!CoolDwarf.opac.aesopus.load@\spxentry{CoolDwarf.opac.aesopus.load}}\index{CoolDwarf.opac.aesopus.load@\spxentry{CoolDwarf.opac.aesopus.load}!module@\spxentry{module}}\index{load\_lowtempopac() (in module CoolDwarf.opac.aesopus.load)@\spxentry{load\_lowtempopac()}\spxextra{in module CoolDwarf.opac.aesopus.load}}

\begin{fulllineitems}
\phantomsection\label{\detokenize{CoolDwarf.opac.aesopus:CoolDwarf.opac.aesopus.load.load_lowtempopac}}
\pysigstartsignatures
\pysiglinewithargsret{\sphinxcode{\sphinxupquote{CoolDwarf.opac.aesopus.load.}}\sphinxbfcode{\sphinxupquote{load\_lowtempopac}}}{\sphinxparam{\DUrole{n}{path}\DUrole{p}{:}\DUrole{w}{ }\DUrole{n}{str}}}{{ $\rightarrow$ dict}}
\pysigstopsignatures
\end{fulllineitems}



\subparagraph{Module contents}
\label{\detokenize{CoolDwarf.opac.aesopus:module-CoolDwarf.opac.aesopus}}\label{\detokenize{CoolDwarf.opac.aesopus:module-contents}}\index{module@\spxentry{module}!CoolDwarf.opac.aesopus@\spxentry{CoolDwarf.opac.aesopus}}\index{CoolDwarf.opac.aesopus@\spxentry{CoolDwarf.opac.aesopus}!module@\spxentry{module}}

\paragraph{Submodules}
\label{\detokenize{CoolDwarf.opac:submodules}}

\paragraph{CoolDwarf.opac.kramer module}
\label{\detokenize{CoolDwarf.opac:module-CoolDwarf.opac.kramer}}\label{\detokenize{CoolDwarf.opac:cooldwarf-opac-kramer-module}}\index{module@\spxentry{module}!CoolDwarf.opac.kramer@\spxentry{CoolDwarf.opac.kramer}}\index{CoolDwarf.opac.kramer@\spxentry{CoolDwarf.opac.kramer}!module@\spxentry{module}}
\sphinxAtStartPar
karmer.py \textendash{} Kramer opacity class for CoolDwarf

\sphinxAtStartPar
This module contains the KramerOpac class, which is used to calculate the Kramer opacity for a given temperature and density.


\subparagraph{Example usage}
\label{\detokenize{CoolDwarf.opac:example-usage}}
\begin{sphinxVerbatim}[commandchars=\\\{\}]
\PYG{g+gp}{\PYGZgt{}\PYGZgt{}\PYGZgt{} }\PYG{k+kn}{from} \PYG{n+nn}{CoolDwarf}\PYG{n+nn}{.}\PYG{n+nn}{opac}\PYG{n+nn}{.}\PYG{n+nn}{kramer} \PYG{k+kn}{import} \PYG{n}{KramerOpac}
\PYG{g+gp}{\PYGZgt{}\PYGZgt{}\PYGZgt{} }\PYG{n}{X}\PYG{p}{,} \PYG{n}{Z} \PYG{o}{=} \PYG{l+m+mf}{0.7}\PYG{p}{,} \PYG{l+m+mf}{0.02}
\PYG{g+gp}{\PYGZgt{}\PYGZgt{}\PYGZgt{} }\PYG{n}{opac} \PYG{o}{=} \PYG{n}{KramerOpac}\PYG{p}{(}\PYG{n}{X}\PYG{p}{,} \PYG{n}{Z}\PYG{p}{)}
\PYG{g+gp}{\PYGZgt{}\PYGZgt{}\PYGZgt{} }\PYG{n}{temp}\PYG{p}{,} \PYG{n}{density} \PYG{o}{=} \PYG{l+m+mf}{1e7}\PYG{p}{,} \PYG{l+m+mf}{1e\PYGZhy{}2}
\PYG{g+gp}{\PYGZgt{}\PYGZgt{}\PYGZgt{} }\PYG{n}{kappa} \PYG{o}{=} \PYG{n}{opac}\PYG{o}{.}\PYG{n}{kappa}\PYG{p}{(}\PYG{n}{temp}\PYG{p}{,} \PYG{n}{density}\PYG{p}{)}
\end{sphinxVerbatim}
\index{KramerOpac (class in CoolDwarf.opac.kramer)@\spxentry{KramerOpac}\spxextra{class in CoolDwarf.opac.kramer}}

\begin{fulllineitems}
\phantomsection\label{\detokenize{CoolDwarf.opac:CoolDwarf.opac.kramer.KramerOpac}}
\pysigstartsignatures
\pysiglinewithargsret{\sphinxbfcode{\sphinxupquote{class\DUrole{w}{ }}}\sphinxcode{\sphinxupquote{CoolDwarf.opac.kramer.}}\sphinxbfcode{\sphinxupquote{KramerOpac}}}{\sphinxparam{\DUrole{n}{X}\DUrole{p}{:}\DUrole{w}{ }\DUrole{n}{float}}\sphinxparamcomma \sphinxparam{\DUrole{n}{Z}\DUrole{p}{:}\DUrole{w}{ }\DUrole{n}{float}}}{}
\pysigstopsignatures
\sphinxAtStartPar
Bases: \sphinxcode{\sphinxupquote{object}}

\sphinxAtStartPar
KramerOpac \textendash{} Kramer opacity class for CoolDwarf

\sphinxAtStartPar
This class is used to calculate the Kramer opacity for a given temperature and density.
\begin{quote}\begin{description}
\sphinxlineitem{Parameters}\begin{description}
\sphinxlineitem{\sphinxstylestrong{X}}{[}float{]}
\sphinxAtStartPar
Hydrogen mass fraction

\sphinxlineitem{\sphinxstylestrong{Z}}{[}float{]}
\sphinxAtStartPar
Metal mass fraction

\end{description}

\end{description}\end{quote}
\subsubsection*{Methods}


\begin{savenotes}\sphinxattablestart
\sphinxthistablewithglobalstyle
\centering
\begin{tabulary}{\linewidth}[t]{TT}
\sphinxtoprule
\sphinxtableatstartofbodyhook
\sphinxAtStartPar
\sphinxstylestrong{kappa(temp, density)}
&
\sphinxAtStartPar
Calculates the Kramer opacity at the given temperature and density
\\
\sphinxbottomrule
\end{tabulary}
\sphinxtableafterendhook\par
\sphinxattableend\end{savenotes}
\index{kappa() (CoolDwarf.opac.kramer.KramerOpac method)@\spxentry{kappa()}\spxextra{CoolDwarf.opac.kramer.KramerOpac method}}

\begin{fulllineitems}
\phantomsection\label{\detokenize{CoolDwarf.opac:CoolDwarf.opac.kramer.KramerOpac.kappa}}
\pysigstartsignatures
\pysiglinewithargsret{\sphinxbfcode{\sphinxupquote{kappa}}}{\sphinxparam{\DUrole{n}{temp}\DUrole{p}{:}\DUrole{w}{ }\DUrole{n}{float}}\sphinxparamcomma \sphinxparam{\DUrole{n}{density}\DUrole{p}{:}\DUrole{w}{ }\DUrole{n}{float}}}{{ $\rightarrow$ float}}
\pysigstopsignatures
\sphinxAtStartPar
Function to calculate the Kramer opacity at the given temperature and density
\begin{quote}\begin{description}
\sphinxlineitem{Parameters}\begin{description}
\sphinxlineitem{\sphinxstylestrong{temp}}{[}float{]}
\sphinxAtStartPar
Temperature in Kelvin

\sphinxlineitem{\sphinxstylestrong{density}}{[}float{]}
\sphinxAtStartPar
Density in g/cm\textasciicircum{}3

\end{description}

\sphinxlineitem{Returns}\begin{description}
\sphinxlineitem{\sphinxstylestrong{kappa}}{[}float{]}
\sphinxAtStartPar
Kramer opacity at the given temperature and density in cm\textasciicircum{}2/g

\end{description}

\end{description}\end{quote}

\end{fulllineitems}


\end{fulllineitems}



\paragraph{CoolDwarf.opac.opacInterp module}
\label{\detokenize{CoolDwarf.opac:module-CoolDwarf.opac.opacInterp}}\label{\detokenize{CoolDwarf.opac:cooldwarf-opac-opacinterp-module}}\index{module@\spxentry{module}!CoolDwarf.opac.opacInterp@\spxentry{CoolDwarf.opac.opacInterp}}\index{CoolDwarf.opac.opacInterp@\spxentry{CoolDwarf.opac.opacInterp}!module@\spxentry{module}}\index{OPACInterp (class in CoolDwarf.opac.opacInterp)@\spxentry{OPACInterp}\spxextra{class in CoolDwarf.opac.opacInterp}}

\begin{fulllineitems}
\phantomsection\label{\detokenize{CoolDwarf.opac:CoolDwarf.opac.opacInterp.OPACInterp}}
\pysigstartsignatures
\pysiglinewithargsret{\sphinxbfcode{\sphinxupquote{class\DUrole{w}{ }}}\sphinxcode{\sphinxupquote{CoolDwarf.opac.opacInterp.}}\sphinxbfcode{\sphinxupquote{OPACInterp}}}{\sphinxparam{\DUrole{n}{opacDict}}\sphinxparamcomma \sphinxparam{\DUrole{n}{X}}\sphinxparamcomma \sphinxparam{\DUrole{n}{Z}}}{}
\pysigstopsignatures
\sphinxAtStartPar
Bases: \sphinxcode{\sphinxupquote{object}}
\subsubsection*{Methods}


\begin{savenotes}\sphinxattablestart
\sphinxthistablewithglobalstyle
\centering
\begin{tabulary}{\linewidth}[t]{TT}
\sphinxtoprule
\sphinxtableatstartofbodyhook
\sphinxAtStartPar
\sphinxstylestrong{kappa}
&\\
\sphinxbottomrule
\end{tabulary}
\sphinxtableafterendhook\par
\sphinxattableend\end{savenotes}
\index{kappa() (CoolDwarf.opac.opacInterp.OPACInterp method)@\spxentry{kappa()}\spxextra{CoolDwarf.opac.opacInterp.OPACInterp method}}

\begin{fulllineitems}
\phantomsection\label{\detokenize{CoolDwarf.opac:CoolDwarf.opac.opacInterp.OPACInterp.kappa}}
\pysigstartsignatures
\pysiglinewithargsret{\sphinxbfcode{\sphinxupquote{kappa}}}{\sphinxparam{\DUrole{n}{temp}}\sphinxparamcomma \sphinxparam{\DUrole{n}{density}}}{}
\pysigstopsignatures
\end{fulllineitems}


\end{fulllineitems}



\paragraph{Module contents}
\label{\detokenize{CoolDwarf.opac:module-CoolDwarf.opac}}\label{\detokenize{CoolDwarf.opac:module-contents}}\index{module@\spxentry{module}!CoolDwarf.opac@\spxentry{CoolDwarf.opac}}\index{CoolDwarf.opac@\spxentry{CoolDwarf.opac}!module@\spxentry{module}}
\sphinxstepscope


\subsubsection{CoolDwarf.star package}
\label{\detokenize{CoolDwarf.star:cooldwarf-star-package}}\label{\detokenize{CoolDwarf.star::doc}}

\paragraph{Submodules}
\label{\detokenize{CoolDwarf.star:submodules}}

\paragraph{CoolDwarf.star.sphere module}
\label{\detokenize{CoolDwarf.star:module-CoolDwarf.star.sphere}}\label{\detokenize{CoolDwarf.star:cooldwarf-star-sphere-module}}\index{module@\spxentry{module}!CoolDwarf.star.sphere@\spxentry{CoolDwarf.star.sphere}}\index{CoolDwarf.star.sphere@\spxentry{CoolDwarf.star.sphere}!module@\spxentry{module}}
\sphinxAtStartPar
sphere.py

\sphinxAtStartPar
This module contains the VoxelSphere class, which represents a 3D voxelized sphere model for a star. 
The VoxelSphere class provides methods for computing various physical properties of the star, 
including its radius, enclosed mass, energy flux, and more.

\sphinxAtStartPar
The VoxelSphere class uses an equation of state (EOS) for the star, which can be inverted to compute 
pressure and temperature grids. It also provides methods for updating the energy of the star and 
recomputing its state.

\sphinxAtStartPar
The module imports several utility functions and constants from the CoolDwarf package, 
as well as classes for handling errors related to energy conservation and non\sphinxhyphen{}convergence.

\sphinxAtStartPar
All units are in non log space cgs unless otherwise specified.


\subparagraph{Dependencies}
\label{\detokenize{CoolDwarf.star:dependencies}}\begin{itemize}
\item {} 
\sphinxAtStartPar
numpy

\item {} 
\sphinxAtStartPar
tqdm

\item {} 
\sphinxAtStartPar
torch

\item {} 
\sphinxAtStartPar
cupy

\item {} 
\sphinxAtStartPar
pandas

\item {} 
\sphinxAtStartPar
scipy.interpolate

\item {} 
\sphinxAtStartPar
CoolDwarf.utils.math

\item {} 
\sphinxAtStartPar
CoolDwarf.utils.const

\item {} 
\sphinxAtStartPar
CoolDwarf.utils.format

\item {} 
\sphinxAtStartPar
CoolDwarf.EOS

\item {} 
\sphinxAtStartPar
CoolDwarf.model

\item {} 
\sphinxAtStartPar
CoolDwarf.err

\end{itemize}


\subparagraph{Classes}
\label{\detokenize{CoolDwarf.star:classes}}\begin{itemize}
\item {} 
\sphinxAtStartPar
VoxelSphere: Represents a 3D voxelized sphere model for a star.

\end{itemize}


\subparagraph{Functions}
\label{\detokenize{CoolDwarf.star:functions}}\begin{itemize}
\item {} 
\sphinxAtStartPar
default\_tol: Returns a dictionary of default numerical tolerances for the cooling model.

\end{itemize}


\subparagraph{Exceptions}
\label{\detokenize{CoolDwarf.star:exceptions}}\begin{itemize}
\item {} 
\sphinxAtStartPar
EnergyConservationError: Raised when there is a violation of energy conservation.

\item {} 
\sphinxAtStartPar
NonConvergenceError: Raised when a computation fails to converge.

\item {} 
\sphinxAtStartPar
VolumeError: Raised when the volume error is greater than the tolerance.

\item {} 
\sphinxAtStartPar
ResolutionError: Raised when the resolution is insufficient.

\end{itemize}


\subparagraph{Example Usage}
\label{\detokenize{CoolDwarf.star:example-usage}}
\begin{sphinxVerbatim}[commandchars=\\\{\}]
\PYG{g+gp}{\PYGZgt{}\PYGZgt{}\PYGZgt{} }\PYG{k+kn}{from} \PYG{n+nn}{CoolDwarf}\PYG{n+nn}{.}\PYG{n+nn}{star} \PYG{k+kn}{import} \PYG{n}{VoxelSphere}
\PYG{g+gp}{\PYGZgt{}\PYGZgt{}\PYGZgt{} }\PYG{k+kn}{from} \PYG{n+nn}{CoolDwarf}\PYG{n+nn}{.}\PYG{n+nn}{utils}\PYG{n+nn}{.}\PYG{n+nn}{plot} \PYG{k+kn}{import} \PYG{n}{plot\PYGZus{}3d\PYGZus{}gradients}\PYG{p}{,} \PYG{n}{visualize\PYGZus{}scalar\PYGZus{}field}
\PYG{g+gp}{\PYGZgt{}\PYGZgt{}\PYGZgt{} }\PYG{k+kn}{from} \PYG{n+nn}{CoolDwarf}\PYG{n+nn}{.}\PYG{n+nn}{utils} \PYG{k+kn}{import} \PYG{n}{setup\PYGZus{}logging}
\PYG{g+gp}{\PYGZgt{}\PYGZgt{}\PYGZgt{} }\PYG{k+kn}{from} \PYG{n+nn}{CoolDwarf}\PYG{n+nn}{.}\PYG{n+nn}{EOS} \PYG{k+kn}{import} \PYG{n}{get\PYGZus{}eos}
\PYG{g+gp}{\PYGZgt{}\PYGZgt{}\PYGZgt{} }\PYG{k+kn}{from} \PYG{n+nn}{CoolDwarf}\PYG{n+nn}{.}\PYG{n+nn}{opac} \PYG{k+kn}{import} \PYG{n}{KramerOpac}
\PYG{g+gp}{\PYGZgt{}\PYGZgt{}\PYGZgt{} }\PYG{k+kn}{from} \PYG{n+nn}{CoolDwarf}\PYG{n+nn}{.}\PYG{n+nn}{EOS}\PYG{n+nn}{.}\PYG{n+nn}{invert} \PYG{k+kn}{import} \PYG{n}{Inverter}
\end{sphinxVerbatim}

\begin{sphinxVerbatim}[commandchars=\\\{\}]
\PYG{g+gp}{\PYGZgt{}\PYGZgt{}\PYGZgt{} }\PYG{k+kn}{import} \PYG{n+nn}{numpy} \PYG{k}{as} \PYG{n+nn}{cp}
\PYG{g+gp}{\PYGZgt{}\PYGZgt{}\PYGZgt{} }\PYG{k+kn}{import} \PYG{n+nn}{matplotlib}\PYG{n+nn}{.}\PYG{n+nn}{pyplot} \PYG{k}{as} \PYG{n+nn}{plt}
\end{sphinxVerbatim}

\begin{sphinxVerbatim}[commandchars=\\\{\}]
\PYG{g+gp}{\PYGZgt{}\PYGZgt{}\PYGZgt{} }\PYG{n}{setup\PYGZus{}logging}\PYG{p}{(}\PYG{n}{debug}\PYG{o}{=}\PYG{k+kc}{True}\PYG{p}{)}
\end{sphinxVerbatim}

\begin{sphinxVerbatim}[commandchars=\\\{\}]
\PYG{g+gp}{\PYGZgt{}\PYGZgt{}\PYGZgt{} }\PYG{n}{EOS} \PYG{o}{=} \PYG{n}{get\PYGZus{}eos}\PYG{p}{(}\PYG{l+s+s2}{\PYGZdq{}}\PYG{l+s+s2}{EOS/TABLEEOS\PYGZus{}2021\PYGZus{}Trho\PYGZus{}Y0292\PYGZus{}v1}\PYG{l+s+s2}{\PYGZdq{}}\PYG{p}{,} \PYG{l+s+s2}{\PYGZdq{}}\PYG{l+s+s2}{CD21}\PYG{l+s+s2}{\PYGZdq{}}\PYG{p}{)}
\PYG{g+gp}{\PYGZgt{}\PYGZgt{}\PYGZgt{} }\PYG{n}{opac} \PYG{o}{=} \PYG{n}{KramerOpac}\PYG{p}{(}\PYG{l+m+mf}{0.7}\PYG{p}{,} \PYG{l+m+mf}{0.02}\PYG{p}{)}
\PYG{g+gp}{\PYGZgt{}\PYGZgt{}\PYGZgt{} }\PYG{n}{sphere} \PYG{o}{=} \PYG{n}{VoxelSphere}\PYG{p}{(}\PYG{l+m+mf}{8e31}\PYG{p}{,} \PYG{l+s+s2}{\PYGZdq{}}\PYG{l+s+s2}{BrownDwarfMESA/BD\PYGZus{}TEST.mod}\PYG{l+s+s2}{\PYGZdq{}}\PYG{p}{,} \PYG{n}{EOS}\PYG{p}{,} \PYG{n}{opac}\PYG{p}{,} \PYG{n}{radialResolution}\PYG{o}{=}\PYG{l+m+mi}{50}\PYG{p}{,} \PYG{n}{altitudinalResolition}\PYG{o}{=}\PYG{l+m+mi}{10}\PYG{p}{,} \PYG{n}{azimuthalResolition}\PYG{o}{=}\PYG{l+m+mi}{20}\PYG{p}{)}
\PYG{g+gp}{\PYGZgt{}\PYGZgt{}\PYGZgt{} }\PYG{n}{sphere}\PYG{o}{.}\PYG{n}{evolve}\PYG{p}{(}\PYG{n}{maxTime}\PYG{o}{=}\PYG{l+m+mf}{3.154e+7}\PYG{p}{,} \PYG{n}{dt}\PYG{o}{=}\PYG{l+m+mi}{86400}\PYG{p}{)}
\PYG{g+gp}{\PYGZgt{}\PYGZgt{}\PYGZgt{} }\PYG{n+nb}{print}\PYG{p}{(}\PYG{l+s+sa}{f}\PYG{l+s+s2}{\PYGZdq{}}\PYG{l+s+s2}{Surface Temp: }\PYG{l+s+si}{\PYGZob{}}\PYG{n}{sphere}\PYG{o}{.}\PYG{n}{surface\PYGZus{}temperature\PYGZus{}profile}\PYG{l+s+si}{\PYGZcb{}}\PYG{l+s+s2}{\PYGZdq{}}\PYG{p}{)}
\end{sphinxVerbatim}
\index{VoxelSphere (class in CoolDwarf.star.sphere)@\spxentry{VoxelSphere}\spxextra{class in CoolDwarf.star.sphere}}

\begin{fulllineitems}
\phantomsection\label{\detokenize{CoolDwarf.star:CoolDwarf.star.sphere.VoxelSphere}}
\pysigstartsignatures
\pysiglinewithargsret{\sphinxbfcode{\sphinxupquote{class\DUrole{w}{ }}}\sphinxcode{\sphinxupquote{CoolDwarf.star.sphere.}}\sphinxbfcode{\sphinxupquote{VoxelSphere}}}{\sphinxparam{\DUrole{n}{mass}}\sphinxparamcomma \sphinxparam{\DUrole{n}{model}}\sphinxparamcomma \sphinxparam{\DUrole{n}{EOS}}\sphinxparamcomma \sphinxparam{\DUrole{n}{opac}}\sphinxparamcomma \sphinxparam{\DUrole{n}{pressureRegularization}\DUrole{o}{=}\DUrole{default_value}{1e\sphinxhyphen{}05}}\sphinxparamcomma \sphinxparam{\DUrole{n}{radialResolution}\DUrole{o}{=}\DUrole{default_value}{10}}\sphinxparamcomma \sphinxparam{\DUrole{n}{azimuthalResolition}\DUrole{o}{=}\DUrole{default_value}{10}}\sphinxparamcomma \sphinxparam{\DUrole{n}{altitudinalResolition}\DUrole{o}{=}\DUrole{default_value}{10}}\sphinxparamcomma \sphinxparam{\DUrole{n}{t0}\DUrole{o}{=}\DUrole{default_value}{0}}\sphinxparamcomma \sphinxparam{\DUrole{n}{X}\DUrole{o}{=}\DUrole{default_value}{0.75}}\sphinxparamcomma \sphinxparam{\DUrole{n}{Y}\DUrole{o}{=}\DUrole{default_value}{0.25}}\sphinxparamcomma \sphinxparam{\DUrole{n}{Z}\DUrole{o}{=}\DUrole{default_value}{0}}\sphinxparamcomma \sphinxparam{\DUrole{n}{tol}\DUrole{o}{=}\DUrole{default_value}{\{\textquotesingle{}maxEChange\textquotesingle{}: 0.0001, \textquotesingle{}relax\textquotesingle{}: 1e\sphinxhyphen{}06, \textquotesingle{}volCheck\textquotesingle{}: 0.01\}}}\sphinxparamcomma \sphinxparam{\DUrole{n}{modelFormat}\DUrole{o}{=}\DUrole{default_value}{\textquotesingle{}mesa\textquotesingle{}}}\sphinxparamcomma \sphinxparam{\DUrole{n}{alpha}\DUrole{o}{=}\DUrole{default_value}{1.901}}\sphinxparamcomma \sphinxparam{\DUrole{n}{mindt}\DUrole{o}{=}\DUrole{default_value}{0.1}}\sphinxparamcomma \sphinxparam{\DUrole{n}{cfl\_factor}\DUrole{o}{=}\DUrole{default_value}{0.5}}\sphinxparamcomma \sphinxparam{\DUrole{n}{imodelOut}\DUrole{o}{=}\DUrole{default_value}{False}}\sphinxparamcomma \sphinxparam{\DUrole{n}{imodelOutCadence}\DUrole{o}{=}\DUrole{default_value}{1000}}\sphinxparamcomma \sphinxparam{\DUrole{n}{imodelOutCadenceUnit}\DUrole{o}{=}\DUrole{default_value}{\textquotesingle{}s\textquotesingle{}}}\sphinxparamcomma \sphinxparam{\DUrole{n}{fmodelOut}\DUrole{o}{=}\DUrole{default_value}{True}}}{}
\pysigstopsignatures
\sphinxAtStartPar
Bases: \sphinxcode{\sphinxupquote{object}}

\sphinxAtStartPar
A class to represent a 3D voxelized sphere model for a star.
\begin{quote}\begin{description}
\sphinxlineitem{Raises}\begin{description}
\sphinxlineitem{EnergyConservationError}
\sphinxAtStartPar
If there is a violation of energy conservation.

\sphinxlineitem{NonConvergenceError}
\sphinxAtStartPar
If a computation fails to converge.

\end{description}

\sphinxlineitem{Attributes}\begin{description}
\sphinxlineitem{\sphinxstylestrong{CONST}}{[}dict{]}
\sphinxAtStartPar
A dictionary of physical constants.

\sphinxlineitem{{\hyperref[\detokenize{CoolDwarf.star:CoolDwarf.star.sphere.VoxelSphere.mass}]{\sphinxcrossref{\sphinxcode{\sphinxupquote{mass}}}}}}{[}float{]}
\sphinxAtStartPar
Returns the mass grid for the star.

\sphinxlineitem{\sphinxstylestrong{model}}{[}str{]}
\sphinxAtStartPar
The name of the stellar model to use.

\sphinxlineitem{\sphinxstylestrong{modelFormat}}{[}str{]}
\sphinxAtStartPar
The format of the stellar model. Default is ‘mesa’. May also be ‘dsep’.

\sphinxlineitem{\sphinxstylestrong{EOS}}{[}EOS{]}
\sphinxAtStartPar
The equation of state for the star.

\sphinxlineitem{\sphinxstylestrong{opac}}{[}Opac{]}
\sphinxAtStartPar
The opacity of the star.

\sphinxlineitem{\sphinxstylestrong{pressureRegularization}}{[}float{]}
\sphinxAtStartPar
A regularization parameter for the pressure computation.

\sphinxlineitem{\sphinxstylestrong{radialResolution}}{[}int{]}
\sphinxAtStartPar
The number of radial divisions in the voxelized sphere.

\sphinxlineitem{\sphinxstylestrong{azimuthalResolution}}{[}int{]}
\sphinxAtStartPar
The number of azimuthal divisions in the voxelized sphere.

\sphinxlineitem{\sphinxstylestrong{t0}}{[}float{]}
\sphinxAtStartPar
The initial time of the star.

\sphinxlineitem{\sphinxstylestrong{X}}{[}float{]}
\sphinxAtStartPar
The hydrogen mass fraction of the star.

\sphinxlineitem{\sphinxstylestrong{Y}}{[}float{]}
\sphinxAtStartPar
The helium mass fraction of the star.

\sphinxlineitem{\sphinxstylestrong{Z}}{[}float{]}
\sphinxAtStartPar
The metal mass fraction of the star.

\sphinxlineitem{\sphinxstylestrong{alpha}}{[}float{]}
\sphinxAtStartPar
The mixing length parameter.

\sphinxlineitem{\sphinxstylestrong{mindt}}{[}float{]}
\sphinxAtStartPar
The minimum timestep for the star.

\sphinxlineitem{\sphinxstylestrong{cfl\_factor}}{[}float{]}
\sphinxAtStartPar
The Courant\sphinxhyphen{}Friedrichs\sphinxhyphen{}Lewy factor for the star.

\sphinxlineitem{\sphinxstylestrong{tol}}{[}dict{]}
\sphinxAtStartPar
A dictionary of numerical tolerances for the cooling model.
Keys are ‘relax’ and ‘maxEChange’. Default is \{‘relax’: 1e\sphinxhyphen{}6, ‘maxEChange’: 1e\sphinxhyphen{}4, ‘volCheck’: 1e\sphinxhyphen{}2\}.
Relax is the relaxation parameter for the energy update. 
MaxEChange is the maximum fractional change in energy allowed per timestep.
volCheck is the maximum fractional error in volume allowed.

\end{description}

\end{description}\end{quote}
\subsubsection*{Methods}


\begin{savenotes}\sphinxattablestart
\sphinxthistablewithglobalstyle
\sphinxthistablewithnovlinesstyle
\centering
\begin{tabulary}{\linewidth}[t]{\X{1}{2}\X{1}{2}}
\sphinxtoprule
\sphinxtableatstartofbodyhook
\sphinxAtStartPar
{\hyperref[\detokenize{CoolDwarf.star:CoolDwarf.star.sphere.VoxelSphere.Cp}]{\sphinxcrossref{\sphinxcode{\sphinxupquote{Cp}}}}}({[}delta\_t{]})
&
\sphinxAtStartPar
Computes the specific heat capacity of the star at constant pressure.
\\
\sphinxhline
\sphinxAtStartPar
{\hyperref[\detokenize{CoolDwarf.star:CoolDwarf.star.sphere.VoxelSphere.as_dict}]{\sphinxcrossref{\sphinxcode{\sphinxupquote{as\_dict}}}}}()
&
\sphinxAtStartPar
Returns a dictionary representation of the star. Returns \sphinxhyphen{}\sphinxhyphen{}\sphinxhyphen{}\sphinxhyphen{}\sphinxhyphen{}\sphinxhyphen{}\sphinxhyphen{}     dict: A dictionary representation of the star.
\\
\sphinxhline
\sphinxAtStartPar
{\hyperref[\detokenize{CoolDwarf.star:CoolDwarf.star.sphere.VoxelSphere.as_pandas}]{\sphinxcrossref{\sphinxcode{\sphinxupquote{as\_pandas}}}}}()
&
\sphinxAtStartPar
Returns a pandas DataFrame representation of the star. Returns \sphinxhyphen{}\sphinxhyphen{}\sphinxhyphen{}\sphinxhyphen{}\sphinxhyphen{}\sphinxhyphen{}\sphinxhyphen{}     pandas.DataFrame: A DataFrame representation of the star.
\\
\sphinxhline
\sphinxAtStartPar
{\hyperref[\detokenize{CoolDwarf.star:CoolDwarf.star.sphere.VoxelSphere.evolve}]{\sphinxcrossref{\sphinxcode{\sphinxupquote{evolve}}}}}({[}maxTime, dt, pbar{]})
&
\sphinxAtStartPar
Evolves the star over a specified time period using a specified timestep.
\\
\sphinxhline
\sphinxAtStartPar
{\hyperref[\detokenize{CoolDwarf.star:CoolDwarf.star.sphere.VoxelSphere.save}]{\sphinxcrossref{\sphinxcode{\sphinxupquote{save}}}}}(filename)
&
\sphinxAtStartPar
Save to a binary file format.
\\
\sphinxhline
\sphinxAtStartPar
{\hyperref[\detokenize{CoolDwarf.star:CoolDwarf.star.sphere.VoxelSphere.spherical_grid_equal_volume}]{\sphinxcrossref{\sphinxcode{\sphinxupquote{spherical\_grid\_equal\_volume}}}}}(numRadial, ...)
&
\sphinxAtStartPar
Generate points within a sphere with equal volume using a stratified sampling approach.
\\
\sphinxhline
\sphinxAtStartPar
{\hyperref[\detokenize{CoolDwarf.star:CoolDwarf.star.sphere.VoxelSphere.timestep}]{\sphinxcrossref{\sphinxcode{\sphinxupquote{timestep}}}}}({[}userdt{]})
&
\sphinxAtStartPar
Computes a timestep for the star based on the CFL condition or the user\sphinxhyphen{}specified timestep.
\\
\sphinxbottomrule
\end{tabulary}
\sphinxtableafterendhook\par
\sphinxattableend\end{savenotes}
\index{CONST (CoolDwarf.star.sphere.VoxelSphere attribute)@\spxentry{CONST}\spxextra{CoolDwarf.star.sphere.VoxelSphere attribute}}

\begin{fulllineitems}
\phantomsection\label{\detokenize{CoolDwarf.star:CoolDwarf.star.sphere.VoxelSphere.CONST}}
\pysigstartsignatures
\pysigline{\sphinxbfcode{\sphinxupquote{CONST}}\sphinxbfcode{\sphinxupquote{\DUrole{w}{ }\DUrole{p}{=}\DUrole{w}{ }\{\textquotesingle{}G\textquotesingle{}: 6.6743e\sphinxhyphen{}08, \textquotesingle{}a\textquotesingle{}: 7.5646e\sphinxhyphen{}15, \textquotesingle{}c\textquotesingle{}: 29979245800.0, \textquotesingle{}mH\textquotesingle{}: 1.00784, \textquotesingle{}mHe\textquotesingle{}: 4.002602\}}}}
\pysigstopsignatures
\end{fulllineitems}

\index{Cp() (CoolDwarf.star.sphere.VoxelSphere method)@\spxentry{Cp()}\spxextra{CoolDwarf.star.sphere.VoxelSphere method}}

\begin{fulllineitems}
\phantomsection\label{\detokenize{CoolDwarf.star:CoolDwarf.star.sphere.VoxelSphere.Cp}}
\pysigstartsignatures
\pysiglinewithargsret{\sphinxbfcode{\sphinxupquote{Cp}}}{\sphinxparam{\DUrole{n}{delta\_t}\DUrole{p}{:}\DUrole{w}{ }\DUrole{n}{float}\DUrole{w}{ }\DUrole{o}{=}\DUrole{w}{ }\DUrole{default_value}{1}}}{}
\pysigstopsignatures
\sphinxAtStartPar
Computes the specific heat capacity of the star at constant pressure.
\begin{quote}\begin{description}
\sphinxlineitem{Parameters}\begin{description}
\sphinxlineitem{\sphinxstylestrong{delta\_t}}{[}float, optional{]}
\sphinxAtStartPar
A small change in temperature for computing the specific heat capacity. Default is 1e\sphinxhyphen{}5.

\end{description}

\sphinxlineitem{Returns}\begin{description}
\sphinxlineitem{xp.ndarray: The specific heat capacity of the star at constant pressure.}
\end{description}

\end{description}\end{quote}

\end{fulllineitems}

\index{as\_dict() (CoolDwarf.star.sphere.VoxelSphere method)@\spxentry{as\_dict()}\spxextra{CoolDwarf.star.sphere.VoxelSphere method}}

\begin{fulllineitems}
\phantomsection\label{\detokenize{CoolDwarf.star:CoolDwarf.star.sphere.VoxelSphere.as_dict}}
\pysigstartsignatures
\pysiglinewithargsret{\sphinxbfcode{\sphinxupquote{as\_dict}}}{}{}
\pysigstopsignatures
\sphinxAtStartPar
Returns a dictionary representation of the star.
Returns
——\sphinxhyphen{}
\begin{quote}

\sphinxAtStartPar
dict: A dictionary representation of the star.
\end{quote}

\end{fulllineitems}

\index{as\_pandas() (CoolDwarf.star.sphere.VoxelSphere method)@\spxentry{as\_pandas()}\spxextra{CoolDwarf.star.sphere.VoxelSphere method}}

\begin{fulllineitems}
\phantomsection\label{\detokenize{CoolDwarf.star:CoolDwarf.star.sphere.VoxelSphere.as_pandas}}
\pysigstartsignatures
\pysiglinewithargsret{\sphinxbfcode{\sphinxupquote{as\_pandas}}}{}{}
\pysigstopsignatures
\sphinxAtStartPar
Returns a pandas DataFrame representation of the star.
Returns
——\sphinxhyphen{}
\begin{quote}

\sphinxAtStartPar
pandas.DataFrame: A DataFrame representation of the star.
\end{quote}

\end{fulllineitems}

\index{cfl\_dt (CoolDwarf.star.sphere.VoxelSphere property)@\spxentry{cfl\_dt}\spxextra{CoolDwarf.star.sphere.VoxelSphere property}}

\begin{fulllineitems}
\phantomsection\label{\detokenize{CoolDwarf.star:CoolDwarf.star.sphere.VoxelSphere.cfl_dt}}
\pysigstartsignatures
\pysigline{\sphinxbfcode{\sphinxupquote{property\DUrole{w}{ }}}\sphinxbfcode{\sphinxupquote{cfl\_dt}}\sphinxbfcode{\sphinxupquote{\DUrole{p}{:}\DUrole{w}{ }float}}}
\pysigstopsignatures
\sphinxAtStartPar
Computes the timestep based on the Courant\sphinxhyphen{}Friedrichs\sphinxhyphen{}Lewy (CFL) condition.
\begin{quote}\begin{description}
\sphinxlineitem{Returns}\begin{description}
\sphinxlineitem{float: The timestep based on the CFL condition.}
\end{description}

\end{description}\end{quote}

\end{fulllineitems}

\index{convective\_energy\_flux (CoolDwarf.star.sphere.VoxelSphere property)@\spxentry{convective\_energy\_flux}\spxextra{CoolDwarf.star.sphere.VoxelSphere property}}

\begin{fulllineitems}
\phantomsection\label{\detokenize{CoolDwarf.star:CoolDwarf.star.sphere.VoxelSphere.convective_energy_flux}}
\pysigstartsignatures
\pysigline{\sphinxbfcode{\sphinxupquote{property\DUrole{w}{ }}}\sphinxbfcode{\sphinxupquote{convective\_energy\_flux}}\sphinxbfcode{\sphinxupquote{\DUrole{p}{:}\DUrole{w}{ }Tuple\DUrole{p}{{[}}ndarray\DUrole{p}{,}\DUrole{w}{ }ndarray\DUrole{p}{,}\DUrole{w}{ }ndarray\DUrole{p}{{]}}}}}
\pysigstopsignatures
\sphinxAtStartPar
Computes the convective energy flux in the radial, azimuthal, and altitudinal directions.
We take a mixing length theory approach to compute the convective energy flux.

\sphinxAtStartPar
The convective enertgy flux for some coordinate direction is given by:
Fconv = (1/2) * rho * Cp * v * (TGrad \sphinxhyphen{} ad)
where rho is the density, Cp is the specific heat capacity, v is the convective velocity along that coordinate axis,
TGrad is the temperature gradient along that coordinte axis, and ad is the adiabatic gradient.
\begin{quote}\begin{description}
\sphinxlineitem{Returns}\begin{description}
\sphinxlineitem{tuple: A tuple of 3D arrays representing the convective energy flux in the radial, azimuthal, and altitudinal directions.}
\sphinxlineitem{The arrays are in the form of (FconvR, FconvTheta, FconvPhi).}
\end{description}

\end{description}\end{quote}

\end{fulllineitems}

\index{convective\_overturn\_timescale (CoolDwarf.star.sphere.VoxelSphere property)@\spxentry{convective\_overturn\_timescale}\spxextra{CoolDwarf.star.sphere.VoxelSphere property}}

\begin{fulllineitems}
\phantomsection\label{\detokenize{CoolDwarf.star:CoolDwarf.star.sphere.VoxelSphere.convective_overturn_timescale}}
\pysigstartsignatures
\pysigline{\sphinxbfcode{\sphinxupquote{property\DUrole{w}{ }}}\sphinxbfcode{\sphinxupquote{convective\_overturn\_timescale}}\sphinxbfcode{\sphinxupquote{\DUrole{p}{:}\DUrole{w}{ }Tuple\DUrole{p}{{[}}ndarray\DUrole{p}{,}\DUrole{w}{ }ndarray\DUrole{p}{,}\DUrole{w}{ }ndarray\DUrole{p}{{]}}}}}
\pysigstopsignatures
\sphinxAtStartPar
Computes the convective overturn timescale in the radial, azimuthal, and altitudinal directions.
The convective overturn timescale is given by the mixing length divided by the convective velocity.
\begin{quote}\begin{description}
\sphinxlineitem{Returns}\begin{description}
\sphinxlineitem{tuple: A tuple of 3D arrays representing the convective overturn timescale in the radial, azimuthal, and altitudinal directions.}
\sphinxlineitem{The arrays are in the form of (tauR, tauTheta, tauPhi).}
\end{description}

\end{description}\end{quote}

\end{fulllineitems}

\index{convective\_velocity (CoolDwarf.star.sphere.VoxelSphere property)@\spxentry{convective\_velocity}\spxextra{CoolDwarf.star.sphere.VoxelSphere property}}

\begin{fulllineitems}
\phantomsection\label{\detokenize{CoolDwarf.star:CoolDwarf.star.sphere.VoxelSphere.convective_velocity}}
\pysigstartsignatures
\pysigline{\sphinxbfcode{\sphinxupquote{property\DUrole{w}{ }}}\sphinxbfcode{\sphinxupquote{convective\_velocity}}\sphinxbfcode{\sphinxupquote{\DUrole{p}{:}\DUrole{w}{ }Tuple\DUrole{p}{{[}}ndarray\DUrole{p}{,}\DUrole{w}{ }ndarray\DUrole{p}{,}\DUrole{w}{ }ndarray\DUrole{p}{{]}}}}}
\pysigstopsignatures
\sphinxAtStartPar
Computes the convective velocity in the radial, azimuthal, and altitudinal directions.
The convective velocity is given by the mixing length divided by two times the square root of the product of the
gravitational acceleration and the difference between the adiabatic gradient and the temperature gradient.
\begin{quote}\begin{description}
\sphinxlineitem{Returns}\begin{description}
\sphinxlineitem{tuple: A tuple of 3D arrays representing the convective velocity in the radial, azimuthal, and altitudinal directions.}
\sphinxlineitem{The arrays are in the form of (vR, vTheta, vPhi).}
\end{description}

\end{description}\end{quote}

\end{fulllineitems}

\index{dEdt (CoolDwarf.star.sphere.VoxelSphere property)@\spxentry{dEdt}\spxextra{CoolDwarf.star.sphere.VoxelSphere property}}

\begin{fulllineitems}
\phantomsection\label{\detokenize{CoolDwarf.star:CoolDwarf.star.sphere.VoxelSphere.dEdt}}
\pysigstartsignatures
\pysigline{\sphinxbfcode{\sphinxupquote{property\DUrole{w}{ }}}\sphinxbfcode{\sphinxupquote{dEdt}}\sphinxbfcode{\sphinxupquote{\DUrole{p}{:}\DUrole{w}{ }ndarray}}}
\pysigstopsignatures
\sphinxAtStartPar
Computes the time derivative of the energy for the star.
\begin{quote}\begin{description}
\sphinxlineitem{Returns}\begin{description}
\sphinxlineitem{xp.ndarray: The time derivative of the energy for the star.}
\end{description}

\end{description}\end{quote}

\end{fulllineitems}

\index{density (CoolDwarf.star.sphere.VoxelSphere property)@\spxentry{density}\spxextra{CoolDwarf.star.sphere.VoxelSphere property}}

\begin{fulllineitems}
\phantomsection\label{\detokenize{CoolDwarf.star:CoolDwarf.star.sphere.VoxelSphere.density}}
\pysigstartsignatures
\pysigline{\sphinxbfcode{\sphinxupquote{property\DUrole{w}{ }}}\sphinxbfcode{\sphinxupquote{density}}\sphinxbfcode{\sphinxupquote{\DUrole{p}{:}\DUrole{w}{ }ndarray}}}
\pysigstopsignatures
\sphinxAtStartPar
Returns the density grid for the star.
Returns
——\sphinxhyphen{}
\begin{quote}

\sphinxAtStartPar
xp.ndarray: The density grid for the star.
\end{quote}

\end{fulllineitems}

\index{enclosed\_mass (CoolDwarf.star.sphere.VoxelSphere property)@\spxentry{enclosed\_mass}\spxextra{CoolDwarf.star.sphere.VoxelSphere property}}

\begin{fulllineitems}
\phantomsection\label{\detokenize{CoolDwarf.star:CoolDwarf.star.sphere.VoxelSphere.enclosed_mass}}
\pysigstartsignatures
\pysigline{\sphinxbfcode{\sphinxupquote{property\DUrole{w}{ }}}\sphinxbfcode{\sphinxupquote{enclosed\_mass}}}
\pysigstopsignatures
\sphinxAtStartPar
Computes the enclosed mass of the star as a function of radius.
\begin{quote}\begin{description}
\sphinxlineitem{Returns}\begin{description}
\sphinxlineitem{interp1d: A 1D interpolation function for the enclosed mass.}
\end{description}

\end{description}\end{quote}

\end{fulllineitems}

\index{energy (CoolDwarf.star.sphere.VoxelSphere property)@\spxentry{energy}\spxextra{CoolDwarf.star.sphere.VoxelSphere property}}

\begin{fulllineitems}
\phantomsection\label{\detokenize{CoolDwarf.star:CoolDwarf.star.sphere.VoxelSphere.energy}}
\pysigstartsignatures
\pysigline{\sphinxbfcode{\sphinxupquote{property\DUrole{w}{ }}}\sphinxbfcode{\sphinxupquote{energy}}\sphinxbfcode{\sphinxupquote{\DUrole{p}{:}\DUrole{w}{ }ndarray}}}
\pysigstopsignatures
\sphinxAtStartPar
Returns the energy grid for the star.

\end{fulllineitems}

\index{energy\_flux (CoolDwarf.star.sphere.VoxelSphere property)@\spxentry{energy\_flux}\spxextra{CoolDwarf.star.sphere.VoxelSphere property}}

\begin{fulllineitems}
\phantomsection\label{\detokenize{CoolDwarf.star:CoolDwarf.star.sphere.VoxelSphere.energy_flux}}
\pysigstartsignatures
\pysigline{\sphinxbfcode{\sphinxupquote{property\DUrole{w}{ }}}\sphinxbfcode{\sphinxupquote{energy\_flux}}\sphinxbfcode{\sphinxupquote{\DUrole{p}{:}\DUrole{w}{ }Tuple\DUrole{p}{{[}}Tuple\DUrole{p}{{[}}ndarray\DUrole{p}{,}\DUrole{w}{ }ndarray\DUrole{p}{,}\DUrole{w}{ }ndarray\DUrole{p}{{]}}\DUrole{p}{,}\DUrole{w}{ }Tuple\DUrole{p}{{[}}ndarray\DUrole{p}{,}\DUrole{w}{ }ndarray\DUrole{p}{,}\DUrole{w}{ }ndarray\DUrole{p}{{]}}\DUrole{p}{{]}}}}}
\pysigstopsignatures
\sphinxAtStartPar
Computes the energy flux in the radial, azimuthal, and altitudinal directions.
\begin{quote}\begin{description}
\sphinxlineitem{Returns}\begin{description}
\sphinxlineitem{tuple: A tuple of 3D arrays representing the energy flux in the radial, azimuthal, and altitudinal directions.}
\sphinxlineitem{The arrays are in the form of ((fluxR, fluxTheta, fluxPhi), (fluxR, fluxTheta, fluxPhi)).}
\end{description}

\end{description}\end{quote}

\end{fulllineitems}

\index{evolve() (CoolDwarf.star.sphere.VoxelSphere method)@\spxentry{evolve()}\spxextra{CoolDwarf.star.sphere.VoxelSphere method}}

\begin{fulllineitems}
\phantomsection\label{\detokenize{CoolDwarf.star:CoolDwarf.star.sphere.VoxelSphere.evolve}}
\pysigstartsignatures
\pysiglinewithargsret{\sphinxbfcode{\sphinxupquote{evolve}}}{\sphinxparam{\DUrole{n}{maxTime}\DUrole{p}{:}\DUrole{w}{ }\DUrole{n}{float}\DUrole{w}{ }\DUrole{o}{=}\DUrole{w}{ }\DUrole{default_value}{31540000.0}}\sphinxparamcomma \sphinxparam{\DUrole{n}{dt}\DUrole{p}{:}\DUrole{w}{ }\DUrole{n}{float}\DUrole{w}{ }\DUrole{o}{=}\DUrole{w}{ }\DUrole{default_value}{86400}}\sphinxparamcomma \sphinxparam{\DUrole{n}{pbar}\DUrole{o}{=}\DUrole{default_value}{False}}}{}
\pysigstopsignatures
\sphinxAtStartPar
Evolves the star over a specified time period using a specified timestep.
\begin{quote}\begin{description}
\sphinxlineitem{Parameters}\begin{description}
\sphinxlineitem{\sphinxstylestrong{maxTime}}{[}float, optional{]}
\sphinxAtStartPar
The maximum time to evolve the star. Default is 3.154e+7.

\sphinxlineitem{\sphinxstylestrong{dt}}{[}float, optional{]}
\sphinxAtStartPar
The timestep to use for the evolution. Default is 86400.

\sphinxlineitem{\sphinxstylestrong{pbar}}{[}bool, optional{]}
\sphinxAtStartPar
Display a progress bar for the evolution. Default is False.

\end{description}

\end{description}\end{quote}

\end{fulllineitems}

\index{flux\_divergence (CoolDwarf.star.sphere.VoxelSphere property)@\spxentry{flux\_divergence}\spxextra{CoolDwarf.star.sphere.VoxelSphere property}}

\begin{fulllineitems}
\phantomsection\label{\detokenize{CoolDwarf.star:CoolDwarf.star.sphere.VoxelSphere.flux_divergence}}
\pysigstartsignatures
\pysigline{\sphinxbfcode{\sphinxupquote{property\DUrole{w}{ }}}\sphinxbfcode{\sphinxupquote{flux\_divergence}}\sphinxbfcode{\sphinxupquote{\DUrole{p}{:}\DUrole{w}{ }Tuple\DUrole{p}{{[}}Tuple\DUrole{p}{{[}}ndarray\DUrole{p}{,}\DUrole{w}{ }ndarray\DUrole{p}{,}\DUrole{w}{ }ndarray\DUrole{p}{{]}}\DUrole{p}{,}\DUrole{w}{ }Tuple\DUrole{p}{{[}}ndarray\DUrole{p}{,}\DUrole{w}{ }ndarray\DUrole{p}{,}\DUrole{w}{ }ndarray\DUrole{p}{{]}}\DUrole{p}{{]}}}}}
\pysigstopsignatures
\sphinxAtStartPar
Computes the divergence of the energy flux in the radial, azimuthal, and altitudinal directions.
\begin{quote}\begin{description}
\sphinxlineitem{Returns}\begin{description}
\sphinxlineitem{tuple: A tuple of 3D arrays representing the divergence of the energy flux in the radial, azimuthal, and altitudinal directions.}
\sphinxlineitem{The arrays are in the form of ((delFConvR, delFConvTheta, delFConvPhi), (delFRadR, delFRadTheta, delFRadPhi)).}
\end{description}

\end{description}\end{quote}

\end{fulllineitems}

\index{gradRadEr (CoolDwarf.star.sphere.VoxelSphere property)@\spxentry{gradRadEr}\spxextra{CoolDwarf.star.sphere.VoxelSphere property}}

\begin{fulllineitems}
\phantomsection\label{\detokenize{CoolDwarf.star:CoolDwarf.star.sphere.VoxelSphere.gradRadEr}}
\pysigstartsignatures
\pysigline{\sphinxbfcode{\sphinxupquote{property\DUrole{w}{ }}}\sphinxbfcode{\sphinxupquote{gradRadEr}}\sphinxbfcode{\sphinxupquote{\DUrole{p}{:}\DUrole{w}{ }Tuple\DUrole{p}{{[}}ndarray\DUrole{p}{,}\DUrole{w}{ }ndarray\DUrole{p}{,}\DUrole{w}{ }ndarray\DUrole{p}{{]}}}}}
\pysigstopsignatures
\sphinxAtStartPar
Computes the radiative energy gradient.
\begin{quote}\begin{description}
\sphinxlineitem{Returns}\begin{description}
\sphinxlineitem{tuple: A tuple of 3D arrays representing the radiative energy gradient in the radial, azimuthal, and altitudinal directions.}
\sphinxlineitem{The arrays are in the form of (delErR, delErTheta, delErPhi).}
\end{description}

\end{description}\end{quote}

\end{fulllineitems}

\index{gradT (CoolDwarf.star.sphere.VoxelSphere property)@\spxentry{gradT}\spxextra{CoolDwarf.star.sphere.VoxelSphere property}}

\begin{fulllineitems}
\phantomsection\label{\detokenize{CoolDwarf.star:CoolDwarf.star.sphere.VoxelSphere.gradT}}
\pysigstartsignatures
\pysigline{\sphinxbfcode{\sphinxupquote{property\DUrole{w}{ }}}\sphinxbfcode{\sphinxupquote{gradT}}\sphinxbfcode{\sphinxupquote{\DUrole{p}{:}\DUrole{w}{ }Tuple\DUrole{p}{{[}}ndarray\DUrole{p}{,}\DUrole{w}{ }ndarray\DUrole{p}{,}\DUrole{w}{ }ndarray\DUrole{p}{{]}}}}}
\pysigstopsignatures
\sphinxAtStartPar
Computes the temperature gradients in the radial, azimuthal, and altitudinal directions.
\begin{quote}\begin{description}
\sphinxlineitem{Returns}\begin{description}
\sphinxlineitem{tuple: A tuple of 3D arrays representing the temperature gradients in the radial, azimuthal, and altitudinal directions.}
\sphinxlineitem{The arrays are in the form of (tGradR, tGradTheta, tGradPhi).}
\end{description}

\end{description}\end{quote}

\end{fulllineitems}

\index{gravitational\_acceleration (CoolDwarf.star.sphere.VoxelSphere property)@\spxentry{gravitational\_acceleration}\spxextra{CoolDwarf.star.sphere.VoxelSphere property}}

\begin{fulllineitems}
\phantomsection\label{\detokenize{CoolDwarf.star:CoolDwarf.star.sphere.VoxelSphere.gravitational_acceleration}}
\pysigstartsignatures
\pysigline{\sphinxbfcode{\sphinxupquote{property\DUrole{w}{ }}}\sphinxbfcode{\sphinxupquote{gravitational\_acceleration}}\sphinxbfcode{\sphinxupquote{\DUrole{p}{:}\DUrole{w}{ }ndarray}}}
\pysigstopsignatures
\sphinxAtStartPar
Computes the gravitational acceleration for the star. If the mass grid is zero at a given grid point,
the gravitational acceleration is set to infinity to deal with the singularity at r=0.
\begin{quote}\begin{description}
\sphinxlineitem{Returns}\begin{description}
\sphinxlineitem{xp.ndarray: The gravitational acceleration for the star.}
\end{description}

\end{description}\end{quote}

\end{fulllineitems}

\index{mass (CoolDwarf.star.sphere.VoxelSphere property)@\spxentry{mass}\spxextra{CoolDwarf.star.sphere.VoxelSphere property}}

\begin{fulllineitems}
\phantomsection\label{\detokenize{CoolDwarf.star:CoolDwarf.star.sphere.VoxelSphere.mass}}
\pysigstartsignatures
\pysigline{\sphinxbfcode{\sphinxupquote{property\DUrole{w}{ }}}\sphinxbfcode{\sphinxupquote{mass}}\sphinxbfcode{\sphinxupquote{\DUrole{p}{:}\DUrole{w}{ }ndarray}}}
\pysigstopsignatures
\sphinxAtStartPar
Returns the mass grid for the star.
Returns
——\sphinxhyphen{}
\begin{quote}

\sphinxAtStartPar
xp.ndarray: The mass grid for the star.
\end{quote}

\end{fulllineitems}

\index{mixing\_length (CoolDwarf.star.sphere.VoxelSphere property)@\spxentry{mixing\_length}\spxextra{CoolDwarf.star.sphere.VoxelSphere property}}

\begin{fulllineitems}
\phantomsection\label{\detokenize{CoolDwarf.star:CoolDwarf.star.sphere.VoxelSphere.mixing_length}}
\pysigstartsignatures
\pysigline{\sphinxbfcode{\sphinxupquote{property\DUrole{w}{ }}}\sphinxbfcode{\sphinxupquote{mixing\_length}}\sphinxbfcode{\sphinxupquote{\DUrole{p}{:}\DUrole{w}{ }ndarray}}}
\pysigstopsignatures
\sphinxAtStartPar
Computes the mixing length for the star.
\begin{quote}\begin{description}
\sphinxlineitem{Returns}\begin{description}
\sphinxlineitem{xp.ndarray: The mixing length for the star.}
\end{description}

\end{description}\end{quote}

\end{fulllineitems}

\index{pressure (CoolDwarf.star.sphere.VoxelSphere property)@\spxentry{pressure}\spxextra{CoolDwarf.star.sphere.VoxelSphere property}}

\begin{fulllineitems}
\phantomsection\label{\detokenize{CoolDwarf.star:CoolDwarf.star.sphere.VoxelSphere.pressure}}
\pysigstartsignatures
\pysigline{\sphinxbfcode{\sphinxupquote{property\DUrole{w}{ }}}\sphinxbfcode{\sphinxupquote{pressure}}\sphinxbfcode{\sphinxupquote{\DUrole{p}{:}\DUrole{w}{ }ndarray}}}
\pysigstopsignatures
\sphinxAtStartPar
Returns the pressure grid for the star.
Returns
——\sphinxhyphen{}
\begin{quote}

\sphinxAtStartPar
xp.ndarray: The pressure grid for the star.
\end{quote}

\end{fulllineitems}

\index{pressure\_scale\_height (CoolDwarf.star.sphere.VoxelSphere property)@\spxentry{pressure\_scale\_height}\spxextra{CoolDwarf.star.sphere.VoxelSphere property}}

\begin{fulllineitems}
\phantomsection\label{\detokenize{CoolDwarf.star:CoolDwarf.star.sphere.VoxelSphere.pressure_scale_height}}
\pysigstartsignatures
\pysigline{\sphinxbfcode{\sphinxupquote{property\DUrole{w}{ }}}\sphinxbfcode{\sphinxupquote{pressure\_scale\_height}}\sphinxbfcode{\sphinxupquote{\DUrole{p}{:}\DUrole{w}{ }ndarray}}}
\pysigstopsignatures
\sphinxAtStartPar
Computes the pressure scale height for the star.
\begin{quote}\begin{description}
\sphinxlineitem{Returns}\begin{description}
\sphinxlineitem{xp.ndarray: The pressure scale height for the star.}
\end{description}

\end{description}\end{quote}

\end{fulllineitems}

\index{radiative\_energy\_flux (CoolDwarf.star.sphere.VoxelSphere property)@\spxentry{radiative\_energy\_flux}\spxextra{CoolDwarf.star.sphere.VoxelSphere property}}

\begin{fulllineitems}
\phantomsection\label{\detokenize{CoolDwarf.star:CoolDwarf.star.sphere.VoxelSphere.radiative_energy_flux}}
\pysigstartsignatures
\pysigline{\sphinxbfcode{\sphinxupquote{property\DUrole{w}{ }}}\sphinxbfcode{\sphinxupquote{radiative\_energy\_flux}}\sphinxbfcode{\sphinxupquote{\DUrole{p}{:}\DUrole{w}{ }Tuple\DUrole{p}{{[}}ndarray\DUrole{p}{,}\DUrole{w}{ }ndarray\DUrole{p}{,}\DUrole{w}{ }ndarray\DUrole{p}{{]}}}}}
\pysigstopsignatures
\sphinxAtStartPar
Computes the radiative energy flux in the radial, azimuthal, and altitudinal directions.
\begin{quote}\begin{description}
\sphinxlineitem{Returns}\begin{description}
\sphinxlineitem{tuple: A tuple of 3D arrays representing the radiative energy flux in the radial, azimuthal, and altitudinal directions.}
\sphinxlineitem{The arrays are in the form of (fluxRadR, fluxRadTheta, fluxRadPhi).}
\end{description}

\end{description}\end{quote}

\end{fulllineitems}

\index{radius (CoolDwarf.star.sphere.VoxelSphere property)@\spxentry{radius}\spxextra{CoolDwarf.star.sphere.VoxelSphere property}}

\begin{fulllineitems}
\phantomsection\label{\detokenize{CoolDwarf.star:CoolDwarf.star.sphere.VoxelSphere.radius}}
\pysigstartsignatures
\pysigline{\sphinxbfcode{\sphinxupquote{property\DUrole{w}{ }}}\sphinxbfcode{\sphinxupquote{radius}}}
\pysigstopsignatures
\sphinxAtStartPar
Returns the radius of the star.
\begin{quote}\begin{description}
\sphinxlineitem{Returns}\begin{description}
\sphinxlineitem{float: The radius of the star.}
\end{description}

\end{description}\end{quote}

\end{fulllineitems}

\index{save() (CoolDwarf.star.sphere.VoxelSphere method)@\spxentry{save()}\spxextra{CoolDwarf.star.sphere.VoxelSphere method}}

\begin{fulllineitems}
\phantomsection\label{\detokenize{CoolDwarf.star:CoolDwarf.star.sphere.VoxelSphere.save}}
\pysigstartsignatures
\pysiglinewithargsret{\sphinxbfcode{\sphinxupquote{save}}}{\sphinxparam{\DUrole{n}{filename}\DUrole{p}{:}\DUrole{w}{ }\DUrole{n}{str}}}{{ $\rightarrow$ bool}}
\pysigstopsignatures
\sphinxAtStartPar
Save to a binary file format. Currently the model format is being defined in the joplin notebook I am using to keep track of development.
\begin{quote}\begin{description}
\sphinxlineitem{Parameters}\begin{description}
\sphinxlineitem{\sphinxstylestrong{filename}}{[}str{]}
\sphinxAtStartPar
The filename to save the star to.

\end{description}

\sphinxlineitem{Returns}\begin{description}
\sphinxlineitem{bool: A flag indicating if the save was successful.}
\end{description}

\end{description}\end{quote}

\end{fulllineitems}

\index{spherical\_grid\_equal\_volume() (CoolDwarf.star.sphere.VoxelSphere method)@\spxentry{spherical\_grid\_equal\_volume()}\spxextra{CoolDwarf.star.sphere.VoxelSphere method}}

\begin{fulllineitems}
\phantomsection\label{\detokenize{CoolDwarf.star:CoolDwarf.star.sphere.VoxelSphere.spherical_grid_equal_volume}}
\pysigstartsignatures
\pysiglinewithargsret{\sphinxbfcode{\sphinxupquote{spherical\_grid\_equal\_volume}}}{\sphinxparam{\DUrole{n}{numRadial}}\sphinxparamcomma \sphinxparam{\DUrole{n}{numTheta}}\sphinxparamcomma \sphinxparam{\DUrole{n}{numPhi}}\sphinxparamcomma \sphinxparam{\DUrole{n}{radius}}}{}
\pysigstopsignatures
\sphinxAtStartPar
Generate points within a sphere with equal volume using a stratified sampling approach.
Returns radius, theta, phi as meshgrids and volume elements for each point.
\begin{quote}\begin{description}
\sphinxlineitem{Parameters}\begin{description}
\sphinxlineitem{\sphinxstylestrong{numRadial}}{[}int{]}
\sphinxAtStartPar
Number of radial segments.

\sphinxlineitem{\sphinxstylestrong{numTheta}}{[}int{]}
\sphinxAtStartPar
Number of azimuthal segments.

\sphinxlineitem{\sphinxstylestrong{numPhi}}{[}int{]}
\sphinxAtStartPar
Number of altitudinal segments.

\sphinxlineitem{\sphinxstylestrong{radius}}{[}float{]}
\sphinxAtStartPar
Radius of the sphere.

\end{description}

\sphinxlineitem{Returns}\begin{description}
\sphinxlineitem{tuple: A tuple of meshgrids for the radial, azimuthal, and altitudinal positions, and the volume elements.}
\end{description}

\sphinxlineitem{Raises}\begin{description}
\sphinxlineitem{VolumeError}
\sphinxAtStartPar
If the volume error is greater than the tolerance.

\end{description}

\end{description}\end{quote}

\end{fulllineitems}

\index{surface\_temperature\_profile (CoolDwarf.star.sphere.VoxelSphere property)@\spxentry{surface\_temperature\_profile}\spxextra{CoolDwarf.star.sphere.VoxelSphere property}}

\begin{fulllineitems}
\phantomsection\label{\detokenize{CoolDwarf.star:CoolDwarf.star.sphere.VoxelSphere.surface_temperature_profile}}
\pysigstartsignatures
\pysigline{\sphinxbfcode{\sphinxupquote{property\DUrole{w}{ }}}\sphinxbfcode{\sphinxupquote{surface\_temperature\_profile}}\sphinxbfcode{\sphinxupquote{\DUrole{p}{:}\DUrole{w}{ }ndarray}}}
\pysigstopsignatures
\sphinxAtStartPar
Computes the surface temperature profile for the star.
\begin{quote}\begin{description}
\sphinxlineitem{Returns}\begin{description}
\sphinxlineitem{xp.ndarray: The surface temperature profile for the star.}
\end{description}

\end{description}\end{quote}

\end{fulllineitems}

\index{temperature (CoolDwarf.star.sphere.VoxelSphere property)@\spxentry{temperature}\spxextra{CoolDwarf.star.sphere.VoxelSphere property}}

\begin{fulllineitems}
\phantomsection\label{\detokenize{CoolDwarf.star:CoolDwarf.star.sphere.VoxelSphere.temperature}}
\pysigstartsignatures
\pysigline{\sphinxbfcode{\sphinxupquote{property\DUrole{w}{ }}}\sphinxbfcode{\sphinxupquote{temperature}}\sphinxbfcode{\sphinxupquote{\DUrole{p}{:}\DUrole{w}{ }ndarray}}}
\pysigstopsignatures
\sphinxAtStartPar
Returns the temperature grid for the star.
\begin{quote}\begin{description}
\sphinxlineitem{Returns}\begin{description}
\sphinxlineitem{xp.ndarray: The temperature grid for the star.}
\end{description}

\end{description}\end{quote}

\end{fulllineitems}

\index{timestep() (CoolDwarf.star.sphere.VoxelSphere method)@\spxentry{timestep()}\spxextra{CoolDwarf.star.sphere.VoxelSphere method}}

\begin{fulllineitems}
\phantomsection\label{\detokenize{CoolDwarf.star:CoolDwarf.star.sphere.VoxelSphere.timestep}}
\pysigstartsignatures
\pysiglinewithargsret{\sphinxbfcode{\sphinxupquote{timestep}}}{\sphinxparam{\DUrole{n}{userdt}\DUrole{p}{:}\DUrole{w}{ }\DUrole{n}{float}\DUrole{w}{ }\DUrole{o}{=}\DUrole{w}{ }\DUrole{default_value}{inf}}}{{ $\rightarrow$ float}}
\pysigstopsignatures
\sphinxAtStartPar
Computes a timestep for the star based on the CFL condition or the user\sphinxhyphen{}specified timestep.
The actual timestep used is the minimum of the CFL timestep and the user\sphinxhyphen{}specified timestep, 
and this will be returned.

\sphinxAtStartPar
The energy of the star is then updated based on the computed timestep. Following this, the energy
is used to invert the EOS and update the temperature and density grids. The energy conservation
is checked, and if the energy change is greater than the maximum energy change tolerance, the timestep
is halved and the energy, temperature, density, and pressure grids are reset to their initial values.

\sphinxAtStartPar
When the energy conservation is satisfied, the evolutionary step is incremented, and the time is updated
based on the the acutal timestep used. The timestep is then returned.
\begin{quote}\begin{description}
\sphinxlineitem{Parameters}\begin{description}
\sphinxlineitem{\sphinxstylestrong{userdt}}{[}float, optional{]}
\sphinxAtStartPar
The user\sphinxhyphen{}specified timestep. Default is xp.inf.

\end{description}

\sphinxlineitem{Returns}\begin{description}
\sphinxlineitem{float: The actual timestep used for the star.}
\end{description}

\sphinxlineitem{Raises}\begin{description}
\sphinxlineitem{EnergyConservationError}
\sphinxAtStartPar
If there is a violation of energy conservation.

\sphinxlineitem{NonConvergenceError}
\sphinxAtStartPar
If the model fails to converge after a certain number of timesteps.

\end{description}

\end{description}\end{quote}
\subsubsection*{Examples}

\begin{sphinxVerbatim}[commandchars=\\\{\}]
\PYG{g+gp}{\PYGZgt{}\PYGZgt{}\PYGZgt{} }\PYG{n}{star} \PYG{o}{=} \PYG{n}{VoxelSphere}\PYG{p}{(}\PYG{o}{.}\PYG{o}{.}\PYG{o}{.}\PYG{p}{)}
\PYG{g+gp}{\PYGZgt{}\PYGZgt{}\PYGZgt{} }\PYG{n}{star}\PYG{o}{.}\PYG{n}{timestep}\PYG{p}{(}\PYG{p}{)}
\end{sphinxVerbatim}

\end{fulllineitems}


\end{fulllineitems}

\index{default\_tol() (in module CoolDwarf.star.sphere)@\spxentry{default\_tol()}\spxextra{in module CoolDwarf.star.sphere}}

\begin{fulllineitems}
\phantomsection\label{\detokenize{CoolDwarf.star:CoolDwarf.star.sphere.default_tol}}
\pysigstartsignatures
\pysiglinewithargsret{\sphinxcode{\sphinxupquote{CoolDwarf.star.sphere.}}\sphinxbfcode{\sphinxupquote{default\_tol}}}{}{}
\pysigstopsignatures
\sphinxAtStartPar
Returns a dictionary of default numerical tolerances for the cooling model.
\begin{quote}\begin{description}
\sphinxlineitem{Returns}\begin{description}
\sphinxlineitem{dict: A dictionary of default numerical tolerances for the cooling model.}
\sphinxlineitem{Keys are ‘relax’, ‘maxEChange’, and ‘volCheck’.}
\end{description}

\end{description}\end{quote}

\end{fulllineitems}



\paragraph{Module contents}
\label{\detokenize{CoolDwarf.star:module-CoolDwarf.star}}\label{\detokenize{CoolDwarf.star:module-contents}}\index{module@\spxentry{module}!CoolDwarf.star@\spxentry{CoolDwarf.star}}\index{CoolDwarf.star@\spxentry{CoolDwarf.star}!module@\spxentry{module}}
\sphinxstepscope


\subsubsection{CoolDwarf.utils package}
\label{\detokenize{CoolDwarf.utils:cooldwarf-utils-package}}\label{\detokenize{CoolDwarf.utils::doc}}

\paragraph{Subpackages}
\label{\detokenize{CoolDwarf.utils:subpackages}}
\sphinxstepscope


\subparagraph{CoolDwarf.utils.const package}
\label{\detokenize{CoolDwarf.utils.const:cooldwarf-utils-const-package}}\label{\detokenize{CoolDwarf.utils.const::doc}}

\subparagraph{Submodules}
\label{\detokenize{CoolDwarf.utils.const:submodules}}

\subparagraph{CoolDwarf.utils.const.const module}
\label{\detokenize{CoolDwarf.utils.const:module-CoolDwarf.utils.const.const}}\label{\detokenize{CoolDwarf.utils.const:cooldwarf-utils-const-const-module}}\index{module@\spxentry{module}!CoolDwarf.utils.const.const@\spxentry{CoolDwarf.utils.const.const}}\index{CoolDwarf.utils.const.const@\spxentry{CoolDwarf.utils.const.const}!module@\spxentry{module}}
\sphinxAtStartPar
const.py \textendash{} Constants for CoolDwarf

\sphinxAtStartPar
This module contains the physical constants used in CoolDwarf.

\sphinxAtStartPar
Constants include:
\sphinxhyphen{} mH: Hydrogen atomic mass in amu (1.00784)
\sphinxhyphen{} mHe: Helium atomic mass in amu (4.002602)
\sphinxhyphen{} c: Speed of light in cgs units (2.99792458e10)
\sphinxhyphen{} a: Radiation constant in cgs units (7.5646e\sphinxhyphen{}15)
\sphinxhyphen{} G: Gravitational constant in cgs units (6.6743e\sphinxhyphen{}8)


\subparagraph{Example usage}
\label{\detokenize{CoolDwarf.utils.const:example-usage}}
\begin{sphinxVerbatim}[commandchars=\\\{\}]
\PYG{g+gp}{\PYGZgt{}\PYGZgt{}\PYGZgt{} }\PYG{k+kn}{from} \PYG{n+nn}{CoolDwarf}\PYG{n+nn}{.}\PYG{n+nn}{utils}\PYG{n+nn}{.}\PYG{n+nn}{const}\PYG{n+nn}{.}\PYG{n+nn}{const} \PYG{k+kn}{import} \PYG{n}{CONST}
\PYG{g+gp}{\PYGZgt{}\PYGZgt{}\PYGZgt{} }\PYG{n+nb}{print}\PYG{p}{(}\PYG{n}{CONST}\PYG{p}{[}\PYG{l+s+s1}{\PYGZsq{}}\PYG{l+s+s1}{mH}\PYG{l+s+s1}{\PYGZsq{}}\PYG{p}{]}\PYG{p}{)}
\end{sphinxVerbatim}


\subparagraph{Module contents}
\label{\detokenize{CoolDwarf.utils.const:module-CoolDwarf.utils.const}}\label{\detokenize{CoolDwarf.utils.const:module-contents}}\index{module@\spxentry{module}!CoolDwarf.utils.const@\spxentry{CoolDwarf.utils.const}}\index{CoolDwarf.utils.const@\spxentry{CoolDwarf.utils.const}!module@\spxentry{module}}
\sphinxstepscope


\subparagraph{CoolDwarf.utils.format package}
\label{\detokenize{CoolDwarf.utils.format:cooldwarf-utils-format-package}}\label{\detokenize{CoolDwarf.utils.format::doc}}

\subparagraph{Submodules}
\label{\detokenize{CoolDwarf.utils.format:submodules}}

\subparagraph{CoolDwarf.utils.format.format module}
\label{\detokenize{CoolDwarf.utils.format:module-CoolDwarf.utils.format.format}}\label{\detokenize{CoolDwarf.utils.format:cooldwarf-utils-format-format-module}}\index{module@\spxentry{module}!CoolDwarf.utils.format.format@\spxentry{CoolDwarf.utils.format.format}}\index{CoolDwarf.utils.format.format@\spxentry{CoolDwarf.utils.format.format}!module@\spxentry{module}}\index{format\_number() (in module CoolDwarf.utils.format.format)@\spxentry{format\_number()}\spxextra{in module CoolDwarf.utils.format.format}}

\begin{fulllineitems}
\phantomsection\label{\detokenize{CoolDwarf.utils.format:CoolDwarf.utils.format.format.format_number}}
\pysigstartsignatures
\pysiglinewithargsret{\sphinxcode{\sphinxupquote{CoolDwarf.utils.format.format.}}\sphinxbfcode{\sphinxupquote{format\_number}}}{\sphinxparam{\DUrole{n}{x}}\sphinxparamcomma \sphinxparam{\DUrole{n}{max\_width}}}{}
\pysigstopsignatures
\end{fulllineitems}

\index{pretty\_print\_3d\_array() (in module CoolDwarf.utils.format.format)@\spxentry{pretty\_print\_3d\_array()}\spxextra{in module CoolDwarf.utils.format.format}}

\begin{fulllineitems}
\phantomsection\label{\detokenize{CoolDwarf.utils.format:CoolDwarf.utils.format.format.pretty_print_3d_array}}
\pysigstartsignatures
\pysiglinewithargsret{\sphinxcode{\sphinxupquote{CoolDwarf.utils.format.format.}}\sphinxbfcode{\sphinxupquote{pretty\_print\_3d\_array}}}{\sphinxparam{\DUrole{n}{array3D}}\sphinxparamcomma \sphinxparam{\DUrole{n}{mask}}\sphinxparamcomma \sphinxparam{\DUrole{n}{decimals}\DUrole{o}{=}\DUrole{default_value}{3}}}{}
\pysigstopsignatures
\end{fulllineitems}



\subparagraph{Module contents}
\label{\detokenize{CoolDwarf.utils.format:module-CoolDwarf.utils.format}}\label{\detokenize{CoolDwarf.utils.format:module-contents}}\index{module@\spxentry{module}!CoolDwarf.utils.format@\spxentry{CoolDwarf.utils.format}}\index{CoolDwarf.utils.format@\spxentry{CoolDwarf.utils.format}!module@\spxentry{module}}
\sphinxstepscope


\subparagraph{CoolDwarf.utils.interp package}
\label{\detokenize{CoolDwarf.utils.interp:cooldwarf-utils-interp-package}}\label{\detokenize{CoolDwarf.utils.interp::doc}}

\subparagraph{Submodules}
\label{\detokenize{CoolDwarf.utils.interp:submodules}}

\subparagraph{CoolDwarf.utils.interp.interpolate module}
\label{\detokenize{CoolDwarf.utils.interp:module-CoolDwarf.utils.interp.interpolate}}\label{\detokenize{CoolDwarf.utils.interp:cooldwarf-utils-interp-interpolate-module}}\index{module@\spxentry{module}!CoolDwarf.utils.interp.interpolate@\spxentry{CoolDwarf.utils.interp.interpolate}}\index{CoolDwarf.utils.interp.interpolate@\spxentry{CoolDwarf.utils.interp.interpolate}!module@\spxentry{module}}\index{find\_closest\_values() (in module CoolDwarf.utils.interp.interpolate)@\spxentry{find\_closest\_values()}\spxextra{in module CoolDwarf.utils.interp.interpolate}}

\begin{fulllineitems}
\phantomsection\label{\detokenize{CoolDwarf.utils.interp:CoolDwarf.utils.interp.interpolate.find_closest_values}}
\pysigstartsignatures
\pysiglinewithargsret{\sphinxcode{\sphinxupquote{CoolDwarf.utils.interp.interpolate.}}\sphinxbfcode{\sphinxupquote{find\_closest\_values}}}{\sphinxparam{\DUrole{n}{numList}}\sphinxparamcomma \sphinxparam{\DUrole{n}{targetValue}}}{}
\pysigstopsignatures
\end{fulllineitems}

\index{linear\_interpolate\_dataframes() (in module CoolDwarf.utils.interp.interpolate)@\spxentry{linear\_interpolate\_dataframes()}\spxextra{in module CoolDwarf.utils.interp.interpolate}}

\begin{fulllineitems}
\phantomsection\label{\detokenize{CoolDwarf.utils.interp:CoolDwarf.utils.interp.interpolate.linear_interpolate_dataframes}}
\pysigstartsignatures
\pysiglinewithargsret{\sphinxcode{\sphinxupquote{CoolDwarf.utils.interp.interpolate.}}\sphinxbfcode{\sphinxupquote{linear\_interpolate\_dataframes}}}{\sphinxparam{\DUrole{n}{df\_dict}}\sphinxparamcomma \sphinxparam{\DUrole{n}{target\_key}}}{}
\pysigstopsignatures
\end{fulllineitems}

\index{linear\_interpolate\_ndarray() (in module CoolDwarf.utils.interp.interpolate)@\spxentry{linear\_interpolate\_ndarray()}\spxextra{in module CoolDwarf.utils.interp.interpolate}}

\begin{fulllineitems}
\phantomsection\label{\detokenize{CoolDwarf.utils.interp:CoolDwarf.utils.interp.interpolate.linear_interpolate_ndarray}}
\pysigstartsignatures
\pysiglinewithargsret{\sphinxcode{\sphinxupquote{CoolDwarf.utils.interp.interpolate.}}\sphinxbfcode{\sphinxupquote{linear\_interpolate\_ndarray}}}{\sphinxparam{\DUrole{n}{arrays}}\sphinxparamcomma \sphinxparam{\DUrole{n}{keys}}\sphinxparamcomma \sphinxparam{\DUrole{n}{target}}}{}
\pysigstopsignatures
\end{fulllineitems}



\subparagraph{Module contents}
\label{\detokenize{CoolDwarf.utils.interp:module-CoolDwarf.utils.interp}}\label{\detokenize{CoolDwarf.utils.interp:module-contents}}\index{module@\spxentry{module}!CoolDwarf.utils.interp@\spxentry{CoolDwarf.utils.interp}}\index{CoolDwarf.utils.interp@\spxentry{CoolDwarf.utils.interp}!module@\spxentry{module}}
\sphinxstepscope


\subparagraph{CoolDwarf.utils.math package}
\label{\detokenize{CoolDwarf.utils.math:cooldwarf-utils-math-package}}\label{\detokenize{CoolDwarf.utils.math::doc}}

\subparagraph{Submodules}
\label{\detokenize{CoolDwarf.utils.math:submodules}}

\subparagraph{CoolDwarf.utils.math.calc module}
\label{\detokenize{CoolDwarf.utils.math:module-CoolDwarf.utils.math.calc}}\label{\detokenize{CoolDwarf.utils.math:cooldwarf-utils-math-calc-module}}\index{module@\spxentry{module}!CoolDwarf.utils.math.calc@\spxentry{CoolDwarf.utils.math.calc}}\index{CoolDwarf.utils.math.calc@\spxentry{CoolDwarf.utils.math.calc}!module@\spxentry{module}}
\sphinxAtStartPar
calc.py \textendash{} Calculus related functions for CoolDwarf

\sphinxAtStartPar
This module contains functions related to calculus that are used in CoolDwarf.

\sphinxAtStartPar
Functions include:
\sphinxhyphen{} partial\_derivative\_x: Function to calculate the partial derivative along the x\sphinxhyphen{}axis
\sphinxhyphen{} compute\_partial\_derivatives: Function to compute the partial derivatives of a scalar field


\subparagraph{Dependencies}
\label{\detokenize{CoolDwarf.utils.math:dependencies}}\begin{itemize}
\item {} 
\sphinxAtStartPar
numpy

\end{itemize}


\subparagraph{Example usage}
\label{\detokenize{CoolDwarf.utils.math:example-usage}}
\begin{sphinxVerbatim}[commandchars=\\\{\}]
\PYG{g+gp}{\PYGZgt{}\PYGZgt{}\PYGZgt{} }\PYG{k+kn}{import} \PYG{n+nn}{numpy} \PYG{k}{as} \PYG{n+nn}{np}
\PYG{g+gp}{\PYGZgt{}\PYGZgt{}\PYGZgt{} }\PYG{k+kn}{from} \PYG{n+nn}{CoolDwarf}\PYG{n+nn}{.}\PYG{n+nn}{utils}\PYG{n+nn}{.}\PYG{n+nn}{math}\PYG{n+nn}{.}\PYG{n+nn}{calc} \PYG{k+kn}{import} \PYG{n}{partial\PYGZus{}derivative\PYGZus{}x}\PYG{p}{,} \PYG{n}{compute\PYGZus{}partial\PYGZus{}derivatives}
\PYG{g+gp}{\PYGZgt{}\PYGZgt{}\PYGZgt{} }\PYG{n}{var} \PYG{o}{=} \PYG{n}{np}\PYG{o}{.}\PYG{n}{random}\PYG{o}{.}\PYG{n}{rand}\PYG{p}{(}\PYG{l+m+mi}{10}\PYG{p}{,} \PYG{l+m+mi}{10}\PYG{p}{,} \PYG{l+m+mi}{10}\PYG{p}{)}
\PYG{g+gp}{\PYGZgt{}\PYGZgt{}\PYGZgt{} }\PYG{n}{dx} \PYG{o}{=} \PYG{l+m+mf}{1.0}
\PYG{g+gp}{\PYGZgt{}\PYGZgt{}\PYGZgt{} }\PYG{n}{partial\PYGZus{}x} \PYG{o}{=} \PYG{n}{partial\PYGZus{}derivative\PYGZus{}x}\PYG{p}{(}\PYG{n}{var}\PYG{p}{,} \PYG{n}{dx}\PYG{p}{)}
\PYG{g+gp}{\PYGZgt{}\PYGZgt{}\PYGZgt{} }\PYG{n}{x} \PYG{o}{=} \PYG{n}{np}\PYG{o}{.}\PYG{n}{linspace}\PYG{p}{(}\PYG{l+m+mi}{0}\PYG{p}{,} \PYG{l+m+mi}{1}\PYG{p}{,} \PYG{l+m+mi}{10}\PYG{p}{)}
\PYG{g+gp}{\PYGZgt{}\PYGZgt{}\PYGZgt{} }\PYG{n}{y} \PYG{o}{=} \PYG{n}{np}\PYG{o}{.}\PYG{n}{linspace}\PYG{p}{(}\PYG{l+m+mi}{0}\PYG{p}{,} \PYG{l+m+mi}{1}\PYG{p}{,} \PYG{l+m+mi}{10}\PYG{p}{)}
\PYG{g+gp}{\PYGZgt{}\PYGZgt{}\PYGZgt{} }\PYG{n}{z} \PYG{o}{=} \PYG{n}{np}\PYG{o}{.}\PYG{n}{linspace}\PYG{p}{(}\PYG{l+m+mi}{0}\PYG{p}{,} \PYG{l+m+mi}{1}\PYG{p}{,} \PYG{l+m+mi}{10}\PYG{p}{)}
\PYG{g+gp}{\PYGZgt{}\PYGZgt{}\PYGZgt{} }\PYG{n}{dfdx}\PYG{p}{,} \PYG{n}{dfdy}\PYG{p}{,} \PYG{n}{dfdz} \PYG{o}{=} \PYG{n}{compute\PYGZus{}partial\PYGZus{}derivatives}\PYG{p}{(}\PYG{n}{var}\PYG{p}{,} \PYG{n}{x}\PYG{p}{,} \PYG{n}{y}\PYG{p}{,} \PYG{n}{z}\PYG{p}{)}
\end{sphinxVerbatim}
\index{compute\_partial\_derivatives() (in module CoolDwarf.utils.math.calc)@\spxentry{compute\_partial\_derivatives()}\spxextra{in module CoolDwarf.utils.math.calc}}

\begin{fulllineitems}
\phantomsection\label{\detokenize{CoolDwarf.utils.math:CoolDwarf.utils.math.calc.compute_partial_derivatives}}
\pysigstartsignatures
\pysiglinewithargsret{\sphinxcode{\sphinxupquote{CoolDwarf.utils.math.calc.}}\sphinxbfcode{\sphinxupquote{compute\_partial\_derivatives}}}{\sphinxparam{\DUrole{n}{scalar\_field}\DUrole{p}{:}\DUrole{w}{ }\DUrole{n}{ndarray}}\sphinxparamcomma \sphinxparam{\DUrole{n}{x}\DUrole{p}{:}\DUrole{w}{ }\DUrole{n}{ndarray}}\sphinxparamcomma \sphinxparam{\DUrole{n}{y}\DUrole{p}{:}\DUrole{w}{ }\DUrole{n}{ndarray}}\sphinxparamcomma \sphinxparam{\DUrole{n}{z}\DUrole{p}{:}\DUrole{w}{ }\DUrole{n}{ndarray}}}{{ $\rightarrow$ Tuple\DUrole{p}{{[}}ndarray\DUrole{p}{,}\DUrole{w}{ }ndarray\DUrole{p}{,}\DUrole{w}{ }ndarray\DUrole{p}{{]}}}}
\pysigstopsignatures
\sphinxAtStartPar
Function to compute the partial derivatives of a scalar field
\begin{quote}\begin{description}
\sphinxlineitem{Parameters}\begin{description}
\sphinxlineitem{\sphinxstylestrong{scalar\_field}}{[}np.ndarray{]}
\sphinxAtStartPar
3D array representing the scalar field

\sphinxlineitem{\sphinxstylestrong{x}}{[}np.ndarray{]}
\sphinxAtStartPar
Array of x\sphinxhyphen{}axis values

\sphinxlineitem{\sphinxstylestrong{y}}{[}np.ndarray{]}
\sphinxAtStartPar
Array of y\sphinxhyphen{}axis values

\sphinxlineitem{\sphinxstylestrong{z}}{[}np.ndarray{]}
\sphinxAtStartPar
Array of z\sphinxhyphen{}axis values

\end{description}

\sphinxlineitem{Returns}\begin{description}
\sphinxlineitem{\sphinxstylestrong{dfdx}}{[}np.ndarray{]}
\sphinxAtStartPar
Array containing the partial derivative along the x\sphinxhyphen{}axis

\sphinxlineitem{\sphinxstylestrong{dfdy}}{[}np.ndarray{]}
\sphinxAtStartPar
Array containing the partial derivative along the y\sphinxhyphen{}axis

\sphinxlineitem{\sphinxstylestrong{dfdz}}{[}np.ndarray{]}
\sphinxAtStartPar
Array containing the partial derivative along the z\sphinxhyphen{}axis

\end{description}

\end{description}\end{quote}

\end{fulllineitems}

\index{partial\_derivative\_x() (in module CoolDwarf.utils.math.calc)@\spxentry{partial\_derivative\_x()}\spxextra{in module CoolDwarf.utils.math.calc}}

\begin{fulllineitems}
\phantomsection\label{\detokenize{CoolDwarf.utils.math:CoolDwarf.utils.math.calc.partial_derivative_x}}
\pysigstartsignatures
\pysiglinewithargsret{\sphinxcode{\sphinxupquote{CoolDwarf.utils.math.calc.}}\sphinxbfcode{\sphinxupquote{partial\_derivative\_x}}}{\sphinxparam{\DUrole{n}{var}\DUrole{p}{:}\DUrole{w}{ }\DUrole{n}{ndarray}}\sphinxparamcomma \sphinxparam{\DUrole{n}{dx}\DUrole{p}{:}\DUrole{w}{ }\DUrole{n}{float}}}{{ $\rightarrow$ ndarray}}
\pysigstopsignatures
\sphinxAtStartPar
Function to calculate the partial derivative along the x\sphinxhyphen{}axis
\begin{quote}\begin{description}
\sphinxlineitem{Parameters}\begin{description}
\sphinxlineitem{\sphinxstylestrong{var}}{[}np.ndarray{]}
\sphinxAtStartPar
Array of values to calculate the partial derivative of

\sphinxlineitem{\sphinxstylestrong{dx}}{[}float{]}
\sphinxAtStartPar
Spacing between the x\sphinxhyphen{}axis points

\end{description}

\sphinxlineitem{Returns}\begin{description}
\sphinxlineitem{\sphinxstylestrong{partial\_x}}{[}np.ndarray{]}
\sphinxAtStartPar
Array containing the partial derivative along the x\sphinxhyphen{}axis

\end{description}

\end{description}\end{quote}

\end{fulllineitems}



\subparagraph{CoolDwarf.utils.math.kernel module}
\label{\detokenize{CoolDwarf.utils.math:module-CoolDwarf.utils.math.kernel}}\label{\detokenize{CoolDwarf.utils.math:cooldwarf-utils-math-kernel-module}}\index{module@\spxentry{module}!CoolDwarf.utils.math.kernel@\spxentry{CoolDwarf.utils.math.kernel}}\index{CoolDwarf.utils.math.kernel@\spxentry{CoolDwarf.utils.math.kernel}!module@\spxentry{module}}\index{make\_3d\_kernels() (in module CoolDwarf.utils.math.kernel)@\spxentry{make\_3d\_kernels()}\spxextra{in module CoolDwarf.utils.math.kernel}}

\begin{fulllineitems}
\phantomsection\label{\detokenize{CoolDwarf.utils.math:CoolDwarf.utils.math.kernel.make_3d_kernels}}
\pysigstartsignatures
\pysiglinewithargsret{\sphinxcode{\sphinxupquote{CoolDwarf.utils.math.kernel.}}\sphinxbfcode{\sphinxupquote{make\_3d\_kernels}}}{}{}
\pysigstopsignatures
\end{fulllineitems}



\subparagraph{Module contents}
\label{\detokenize{CoolDwarf.utils.math:module-CoolDwarf.utils.math}}\label{\detokenize{CoolDwarf.utils.math:module-contents}}\index{module@\spxentry{module}!CoolDwarf.utils.math@\spxentry{CoolDwarf.utils.math}}\index{CoolDwarf.utils.math@\spxentry{CoolDwarf.utils.math}!module@\spxentry{module}}
\sphinxstepscope


\subparagraph{CoolDwarf.utils.misc package}
\label{\detokenize{CoolDwarf.utils.misc:cooldwarf-utils-misc-package}}\label{\detokenize{CoolDwarf.utils.misc::doc}}

\subparagraph{Submodules}
\label{\detokenize{CoolDwarf.utils.misc:submodules}}

\subparagraph{CoolDwarf.utils.misc.evolve module}
\label{\detokenize{CoolDwarf.utils.misc:cooldwarf-utils-misc-evolve-module}}

\subparagraph{CoolDwarf.utils.misc.logging module}
\label{\detokenize{CoolDwarf.utils.misc:module-CoolDwarf.utils.misc.logging}}\label{\detokenize{CoolDwarf.utils.misc:cooldwarf-utils-misc-logging-module}}\index{module@\spxentry{module}!CoolDwarf.utils.misc.logging@\spxentry{CoolDwarf.utils.misc.logging}}\index{CoolDwarf.utils.misc.logging@\spxentry{CoolDwarf.utils.misc.logging}!module@\spxentry{module}}
\sphinxAtStartPar
logging.py \textendash{} Logging setup for CoolDwarf

\sphinxAtStartPar
This module contains the setup\_logging function, which is used to set up the logging configuration for CoolDwarf.


\subparagraph{Example usage}
\label{\detokenize{CoolDwarf.utils.misc:example-usage}}
\begin{sphinxVerbatim}[commandchars=\\\{\}]
\PYG{g+gp}{\PYGZgt{}\PYGZgt{}\PYGZgt{} }\PYG{k+kn}{from} \PYG{n+nn}{CoolDwarf}\PYG{n+nn}{.}\PYG{n+nn}{utils}\PYG{n+nn}{.}\PYG{n+nn}{misc}\PYG{n+nn}{.}\PYG{n+nn}{logging} \PYG{k+kn}{import} \PYG{n}{setup\PYGZus{}logging}
\PYG{g+gp}{\PYGZgt{}\PYGZgt{}\PYGZgt{} }\PYG{n}{setup\PYGZus{}logging}\PYG{p}{(}\PYG{n}{debug}\PYG{o}{=}\PYG{k+kc}{True}\PYG{p}{)}
\end{sphinxVerbatim}
\index{CustomFilter (class in CoolDwarf.utils.misc.logging)@\spxentry{CustomFilter}\spxextra{class in CoolDwarf.utils.misc.logging}}

\begin{fulllineitems}
\phantomsection\label{\detokenize{CoolDwarf.utils.misc:CoolDwarf.utils.misc.logging.CustomFilter}}
\pysigstartsignatures
\pysiglinewithargsret{\sphinxbfcode{\sphinxupquote{class\DUrole{w}{ }}}\sphinxcode{\sphinxupquote{CoolDwarf.utils.misc.logging.}}\sphinxbfcode{\sphinxupquote{CustomFilter}}}{\sphinxparam{\DUrole{n}{name}\DUrole{o}{=}\DUrole{default_value}{\textquotesingle{}\textquotesingle{}}}}{}
\pysigstopsignatures
\sphinxAtStartPar
Bases: \sphinxcode{\sphinxupquote{Filter}}
\subsubsection*{Methods}


\begin{savenotes}\sphinxattablestart
\sphinxthistablewithglobalstyle
\sphinxthistablewithnovlinesstyle
\centering
\begin{tabulary}{\linewidth}[t]{\X{1}{2}\X{1}{2}}
\sphinxtoprule
\sphinxtableatstartofbodyhook
\sphinxAtStartPar
{\hyperref[\detokenize{CoolDwarf.utils.misc:CoolDwarf.utils.misc.logging.CustomFilter.filter}]{\sphinxcrossref{\sphinxcode{\sphinxupquote{filter}}}}}(record)
&
\sphinxAtStartPar
Determine if the specified record is to be logged.
\\
\sphinxbottomrule
\end{tabulary}
\sphinxtableafterendhook\par
\sphinxattableend\end{savenotes}
\index{filter() (CoolDwarf.utils.misc.logging.CustomFilter method)@\spxentry{filter()}\spxextra{CoolDwarf.utils.misc.logging.CustomFilter method}}

\begin{fulllineitems}
\phantomsection\label{\detokenize{CoolDwarf.utils.misc:CoolDwarf.utils.misc.logging.CustomFilter.filter}}
\pysigstartsignatures
\pysiglinewithargsret{\sphinxbfcode{\sphinxupquote{filter}}}{\sphinxparam{\DUrole{n}{record}}}{}
\pysigstopsignatures
\sphinxAtStartPar
Determine if the specified record is to be logged.

\sphinxAtStartPar
Returns True if the record should be logged, or False otherwise.
If deemed appropriate, the record may be modified in\sphinxhyphen{}place.

\end{fulllineitems}


\end{fulllineitems}

\index{ExcludeCustomFilter (class in CoolDwarf.utils.misc.logging)@\spxentry{ExcludeCustomFilter}\spxextra{class in CoolDwarf.utils.misc.logging}}

\begin{fulllineitems}
\phantomsection\label{\detokenize{CoolDwarf.utils.misc:CoolDwarf.utils.misc.logging.ExcludeCustomFilter}}
\pysigstartsignatures
\pysiglinewithargsret{\sphinxbfcode{\sphinxupquote{class\DUrole{w}{ }}}\sphinxcode{\sphinxupquote{CoolDwarf.utils.misc.logging.}}\sphinxbfcode{\sphinxupquote{ExcludeCustomFilter}}}{\sphinxparam{\DUrole{n}{name}\DUrole{o}{=}\DUrole{default_value}{\textquotesingle{}\textquotesingle{}}}}{}
\pysigstopsignatures
\sphinxAtStartPar
Bases: \sphinxcode{\sphinxupquote{Filter}}
\subsubsection*{Methods}


\begin{savenotes}\sphinxattablestart
\sphinxthistablewithglobalstyle
\sphinxthistablewithnovlinesstyle
\centering
\begin{tabulary}{\linewidth}[t]{\X{1}{2}\X{1}{2}}
\sphinxtoprule
\sphinxtableatstartofbodyhook
\sphinxAtStartPar
{\hyperref[\detokenize{CoolDwarf.utils.misc:CoolDwarf.utils.misc.logging.ExcludeCustomFilter.filter}]{\sphinxcrossref{\sphinxcode{\sphinxupquote{filter}}}}}(record)
&
\sphinxAtStartPar
Determine if the specified record is to be logged.
\\
\sphinxbottomrule
\end{tabulary}
\sphinxtableafterendhook\par
\sphinxattableend\end{savenotes}
\index{filter() (CoolDwarf.utils.misc.logging.ExcludeCustomFilter method)@\spxentry{filter()}\spxextra{CoolDwarf.utils.misc.logging.ExcludeCustomFilter method}}

\begin{fulllineitems}
\phantomsection\label{\detokenize{CoolDwarf.utils.misc:CoolDwarf.utils.misc.logging.ExcludeCustomFilter.filter}}
\pysigstartsignatures
\pysiglinewithargsret{\sphinxbfcode{\sphinxupquote{filter}}}{\sphinxparam{\DUrole{n}{record}}}{}
\pysigstopsignatures
\sphinxAtStartPar
Determine if the specified record is to be logged.

\sphinxAtStartPar
Returns True if the record should be logged, or False otherwise.
If deemed appropriate, the record may be modified in\sphinxhyphen{}place.

\end{fulllineitems}


\end{fulllineitems}

\index{evolve\_log() (in module CoolDwarf.utils.misc.logging)@\spxentry{evolve\_log()}\spxextra{in module CoolDwarf.utils.misc.logging}}

\begin{fulllineitems}
\phantomsection\label{\detokenize{CoolDwarf.utils.misc:CoolDwarf.utils.misc.logging.evolve_log}}
\pysigstartsignatures
\pysiglinewithargsret{\sphinxcode{\sphinxupquote{CoolDwarf.utils.misc.logging.}}\sphinxbfcode{\sphinxupquote{evolve\_log}}}{\sphinxparam{\DUrole{n}{self}}\sphinxparamcomma \sphinxparam{\DUrole{n}{message}}\sphinxparamcomma \sphinxparam{\DUrole{o}{*}\DUrole{n}{args}}\sphinxparamcomma \sphinxparam{\DUrole{o}{**}\DUrole{n}{kwargs}}}{}
\pysigstopsignatures
\end{fulllineitems}

\index{setup\_logging() (in module CoolDwarf.utils.misc.logging)@\spxentry{setup\_logging()}\spxextra{in module CoolDwarf.utils.misc.logging}}

\begin{fulllineitems}
\phantomsection\label{\detokenize{CoolDwarf.utils.misc:CoolDwarf.utils.misc.logging.setup_logging}}
\pysigstartsignatures
\pysiglinewithargsret{\sphinxcode{\sphinxupquote{CoolDwarf.utils.misc.logging.}}\sphinxbfcode{\sphinxupquote{setup\_logging}}}{\sphinxparam{\DUrole{n}{debug}\DUrole{p}{:}\DUrole{w}{ }\DUrole{n}{bool}\DUrole{w}{ }\DUrole{o}{=}\DUrole{w}{ }\DUrole{default_value}{False}}}{}
\pysigstopsignatures
\sphinxAtStartPar
This function is used to set up the logging configuration for CoolDwarf.
\begin{quote}\begin{description}
\sphinxlineitem{Parameters}\begin{description}
\sphinxlineitem{\sphinxstylestrong{debug}}{[}bool, default=False{]}
\sphinxAtStartPar
If True, sets the logging level to DEBUG. Otherwise, sets the logging level to INFO.

\end{description}

\end{description}\end{quote}

\end{fulllineitems}



\subparagraph{CoolDwarf.utils.misc.ndarray module}
\label{\detokenize{CoolDwarf.utils.misc:module-CoolDwarf.utils.misc.ndarray}}\label{\detokenize{CoolDwarf.utils.misc:cooldwarf-utils-misc-ndarray-module}}\index{module@\spxentry{module}!CoolDwarf.utils.misc.ndarray@\spxentry{CoolDwarf.utils.misc.ndarray}}\index{CoolDwarf.utils.misc.ndarray@\spxentry{CoolDwarf.utils.misc.ndarray}!module@\spxentry{module}}\index{CallbackNDArray (class in CoolDwarf.utils.misc.ndarray)@\spxentry{CallbackNDArray}\spxextra{class in CoolDwarf.utils.misc.ndarray}}

\begin{fulllineitems}
\phantomsection\label{\detokenize{CoolDwarf.utils.misc:CoolDwarf.utils.misc.ndarray.CallbackNDArray}}
\pysigstartsignatures
\pysiglinewithargsret{\sphinxbfcode{\sphinxupquote{class\DUrole{w}{ }}}\sphinxcode{\sphinxupquote{CoolDwarf.utils.misc.ndarray.}}\sphinxbfcode{\sphinxupquote{CallbackNDArray}}}{\sphinxparam{\DUrole{n}{input\_array}}\sphinxparamcomma \sphinxparam{\DUrole{n}{callback}\DUrole{o}{=}\DUrole{default_value}{None}}\sphinxparamcomma \sphinxparam{\DUrole{o}{*}\DUrole{n}{args}}\sphinxparamcomma \sphinxparam{\DUrole{o}{**}\DUrole{n}{kwargs}}}{}
\pysigstopsignatures
\sphinxAtStartPar
Bases: \sphinxcode{\sphinxupquote{ndarray}}
\begin{quote}\begin{description}
\sphinxlineitem{Attributes}\begin{description}
\sphinxlineitem{\sphinxcode{\sphinxupquote{T}}}
\sphinxAtStartPar
View of the transposed array.

\sphinxlineitem{\sphinxcode{\sphinxupquote{base}}}
\sphinxAtStartPar
Base object if memory is from some other object.

\sphinxlineitem{\sphinxcode{\sphinxupquote{ctypes}}}
\sphinxAtStartPar
An object to simplify the interaction of the array with the ctypes module.

\sphinxlineitem{\sphinxcode{\sphinxupquote{data}}}
\sphinxAtStartPar
Python buffer object pointing to the start of the array’s data.

\sphinxlineitem{\sphinxcode{\sphinxupquote{dtype}}}
\sphinxAtStartPar
Data\sphinxhyphen{}type of the array’s elements.

\sphinxlineitem{\sphinxcode{\sphinxupquote{flags}}}
\sphinxAtStartPar
Information about the memory layout of the array.

\sphinxlineitem{\sphinxcode{\sphinxupquote{flat}}}
\sphinxAtStartPar
A 1\sphinxhyphen{}D iterator over the array.

\sphinxlineitem{\sphinxcode{\sphinxupquote{imag}}}
\sphinxAtStartPar
The imaginary part of the array.

\sphinxlineitem{\sphinxcode{\sphinxupquote{itemsize}}}
\sphinxAtStartPar
Length of one array element in bytes.

\sphinxlineitem{\sphinxcode{\sphinxupquote{nbytes}}}
\sphinxAtStartPar
Total bytes consumed by the elements of the array.

\sphinxlineitem{\sphinxcode{\sphinxupquote{ndim}}}
\sphinxAtStartPar
Number of array dimensions.

\sphinxlineitem{\sphinxcode{\sphinxupquote{real}}}
\sphinxAtStartPar
The real part of the array.

\sphinxlineitem{\sphinxcode{\sphinxupquote{shape}}}
\sphinxAtStartPar
Tuple of array dimensions.

\sphinxlineitem{\sphinxcode{\sphinxupquote{size}}}
\sphinxAtStartPar
Number of elements in the array.

\sphinxlineitem{\sphinxcode{\sphinxupquote{strides}}}
\sphinxAtStartPar
Tuple of bytes to step in each dimension when traversing an array.

\end{description}

\end{description}\end{quote}
\subsubsection*{Methods}


\begin{savenotes}
\sphinxatlongtablestart
\sphinxthistablewithglobalstyle
\sphinxthistablewithnovlinesstyle
\makeatletter
  \LTleft \@totalleftmargin plus1fill
  \LTright\dimexpr\columnwidth-\@totalleftmargin-\linewidth\relax plus1fill
\makeatother
\begin{longtable}{\X{1}{2}\X{1}{2}}
\sphinxtoprule
\endfirsthead

\multicolumn{2}{c}{\sphinxnorowcolor
    \makebox[0pt]{\sphinxtablecontinued{\tablename\ \thetable{} \textendash{} continued from previous page}}%
}\\
\sphinxtoprule
\endhead

\sphinxbottomrule
\multicolumn{2}{r}{\sphinxnorowcolor
    \makebox[0pt][r]{\sphinxtablecontinued{continues on next page}}%
}\\
\endfoot

\endlastfoot
\sphinxtableatstartofbodyhook

\sphinxAtStartPar
\sphinxcode{\sphinxupquote{all}}({[}axis, out, keepdims, where{]})
&
\sphinxAtStartPar
Returns True if all elements evaluate to True.
\\
\sphinxhline
\sphinxAtStartPar
\sphinxcode{\sphinxupquote{any}}({[}axis, out, keepdims, where{]})
&
\sphinxAtStartPar
Returns True if any of the elements of \sphinxtitleref{a} evaluate to True.
\\
\sphinxhline
\sphinxAtStartPar
\sphinxcode{\sphinxupquote{argmax}}({[}axis, out, keepdims{]})
&
\sphinxAtStartPar
Return indices of the maximum values along the given axis.
\\
\sphinxhline
\sphinxAtStartPar
\sphinxcode{\sphinxupquote{argmin}}({[}axis, out, keepdims{]})
&
\sphinxAtStartPar
Return indices of the minimum values along the given axis.
\\
\sphinxhline
\sphinxAtStartPar
\sphinxcode{\sphinxupquote{argpartition}}(kth{[}, axis, kind, order{]})
&
\sphinxAtStartPar
Returns the indices that would partition this array.
\\
\sphinxhline
\sphinxAtStartPar
\sphinxcode{\sphinxupquote{argsort}}({[}axis, kind, order{]})
&
\sphinxAtStartPar
Returns the indices that would sort this array.
\\
\sphinxhline
\sphinxAtStartPar
\sphinxcode{\sphinxupquote{astype}}(dtype{[}, order, casting, subok, copy{]})
&
\sphinxAtStartPar
Copy of the array, cast to a specified type.
\\
\sphinxhline
\sphinxAtStartPar
\sphinxcode{\sphinxupquote{byteswap}}({[}inplace{]})
&
\sphinxAtStartPar
Swap the bytes of the array elements
\\
\sphinxhline
\sphinxAtStartPar
\sphinxcode{\sphinxupquote{choose}}(choices{[}, out, mode{]})
&
\sphinxAtStartPar
Use an index array to construct a new array from a set of choices.
\\
\sphinxhline
\sphinxAtStartPar
\sphinxcode{\sphinxupquote{clip}}({[}min, max, out{]})
&
\sphinxAtStartPar
Return an array whose values are limited to \sphinxcode{\sphinxupquote{{[}min, max{]}}}.
\\
\sphinxhline
\sphinxAtStartPar
\sphinxcode{\sphinxupquote{compress}}(condition{[}, axis, out{]})
&
\sphinxAtStartPar
Return selected slices of this array along given axis.
\\
\sphinxhline
\sphinxAtStartPar
\sphinxcode{\sphinxupquote{conj}}()
&
\sphinxAtStartPar
Complex\sphinxhyphen{}conjugate all elements.
\\
\sphinxhline
\sphinxAtStartPar
\sphinxcode{\sphinxupquote{conjugate}}()
&
\sphinxAtStartPar
Return the complex conjugate, element\sphinxhyphen{}wise.
\\
\sphinxhline
\sphinxAtStartPar
\sphinxcode{\sphinxupquote{copy}}({[}order{]})
&
\sphinxAtStartPar
Return a copy of the array.
\\
\sphinxhline
\sphinxAtStartPar
\sphinxcode{\sphinxupquote{cumprod}}({[}axis, dtype, out{]})
&
\sphinxAtStartPar
Return the cumulative product of the elements along the given axis.
\\
\sphinxhline
\sphinxAtStartPar
\sphinxcode{\sphinxupquote{cumsum}}({[}axis, dtype, out{]})
&
\sphinxAtStartPar
Return the cumulative sum of the elements along the given axis.
\\
\sphinxhline
\sphinxAtStartPar
\sphinxcode{\sphinxupquote{diagonal}}({[}offset, axis1, axis2{]})
&
\sphinxAtStartPar
Return specified diagonals.
\\
\sphinxhline
\sphinxAtStartPar
\sphinxcode{\sphinxupquote{dump}}(file)
&
\sphinxAtStartPar
Dump a pickle of the array to the specified file.
\\
\sphinxhline
\sphinxAtStartPar
\sphinxcode{\sphinxupquote{dumps}}()
&
\sphinxAtStartPar
Returns the pickle of the array as a string.
\\
\sphinxhline
\sphinxAtStartPar
\sphinxcode{\sphinxupquote{fill}}(value)
&
\sphinxAtStartPar
Fill the array with a scalar value.
\\
\sphinxhline
\sphinxAtStartPar
\sphinxcode{\sphinxupquote{flatten}}({[}order{]})
&
\sphinxAtStartPar
Return a copy of the array collapsed into one dimension.
\\
\sphinxhline
\sphinxAtStartPar
\sphinxcode{\sphinxupquote{getfield}}(dtype{[}, offset{]})
&
\sphinxAtStartPar
Returns a field of the given array as a certain type.
\\
\sphinxhline
\sphinxAtStartPar
\sphinxcode{\sphinxupquote{item}}(*args)
&
\sphinxAtStartPar
Copy an element of an array to a standard Python scalar and return it.
\\
\sphinxhline
\sphinxAtStartPar
\sphinxcode{\sphinxupquote{itemset}}(*args)
&
\sphinxAtStartPar
Insert scalar into an array (scalar is cast to array\textquotesingle{}s dtype, if possible)
\\
\sphinxhline
\sphinxAtStartPar
\sphinxcode{\sphinxupquote{max}}({[}axis, out, keepdims, initial, where{]})
&
\sphinxAtStartPar
Return the maximum along a given axis.
\\
\sphinxhline
\sphinxAtStartPar
\sphinxcode{\sphinxupquote{mean}}({[}axis, dtype, out, keepdims, where{]})
&
\sphinxAtStartPar
Returns the average of the array elements along given axis.
\\
\sphinxhline
\sphinxAtStartPar
\sphinxcode{\sphinxupquote{min}}({[}axis, out, keepdims, initial, where{]})
&
\sphinxAtStartPar
Return the minimum along a given axis.
\\
\sphinxhline
\sphinxAtStartPar
\sphinxcode{\sphinxupquote{newbyteorder}}({[}new\_order{]})
&
\sphinxAtStartPar
Return the array with the same data viewed with a different byte order.
\\
\sphinxhline
\sphinxAtStartPar
\sphinxcode{\sphinxupquote{nonzero}}()
&
\sphinxAtStartPar
Return the indices of the elements that are non\sphinxhyphen{}zero.
\\
\sphinxhline
\sphinxAtStartPar
\sphinxcode{\sphinxupquote{partition}}(kth{[}, axis, kind, order{]})
&
\sphinxAtStartPar
Rearranges the elements in the array in such a way that the value of the element in kth position is in the position it would be in a sorted array.
\\
\sphinxhline
\sphinxAtStartPar
\sphinxcode{\sphinxupquote{prod}}({[}axis, dtype, out, keepdims, initial, ...{]})
&
\sphinxAtStartPar
Return the product of the array elements over the given axis
\\
\sphinxhline
\sphinxAtStartPar
\sphinxcode{\sphinxupquote{ptp}}({[}axis, out, keepdims{]})
&
\sphinxAtStartPar
Peak to peak (maximum \sphinxhyphen{} minimum) value along a given axis.
\\
\sphinxhline
\sphinxAtStartPar
\sphinxcode{\sphinxupquote{put}}(indices, values{[}, mode{]})
&
\sphinxAtStartPar
Set \sphinxcode{\sphinxupquote{a.flat{[}n{]} = values{[}n{]}}} for all \sphinxtitleref{n} in indices.
\\
\sphinxhline
\sphinxAtStartPar
\sphinxcode{\sphinxupquote{ravel}}({[}order{]})
&
\sphinxAtStartPar
Return a flattened array.
\\
\sphinxhline
\sphinxAtStartPar
\sphinxcode{\sphinxupquote{repeat}}(repeats{[}, axis{]})
&
\sphinxAtStartPar
Repeat elements of an array.
\\
\sphinxhline
\sphinxAtStartPar
\sphinxcode{\sphinxupquote{reshape}}(shape{[}, order{]})
&
\sphinxAtStartPar
Returns an array containing the same data with a new shape.
\\
\sphinxhline
\sphinxAtStartPar
\sphinxcode{\sphinxupquote{resize}}(new\_shape{[}, refcheck{]})
&
\sphinxAtStartPar
Change shape and size of array in\sphinxhyphen{}place.
\\
\sphinxhline
\sphinxAtStartPar
\sphinxcode{\sphinxupquote{round}}({[}decimals, out{]})
&
\sphinxAtStartPar
Return \sphinxtitleref{a} with each element rounded to the given number of decimals.
\\
\sphinxhline
\sphinxAtStartPar
\sphinxcode{\sphinxupquote{searchsorted}}(v{[}, side, sorter{]})
&
\sphinxAtStartPar
Find indices where elements of v should be inserted in a to maintain order.
\\
\sphinxhline
\sphinxAtStartPar
\sphinxcode{\sphinxupquote{setfield}}(val, dtype{[}, offset{]})
&
\sphinxAtStartPar
Put a value into a specified place in a field defined by a data\sphinxhyphen{}type.
\\
\sphinxhline
\sphinxAtStartPar
\sphinxcode{\sphinxupquote{setflags}}({[}write, align, uic{]})
&
\sphinxAtStartPar
Set array flags WRITEABLE, ALIGNED, WRITEBACKIFCOPY, respectively.
\\
\sphinxhline
\sphinxAtStartPar
\sphinxcode{\sphinxupquote{sort}}({[}axis, kind, order{]})
&
\sphinxAtStartPar
Sort an array in\sphinxhyphen{}place.
\\
\sphinxhline
\sphinxAtStartPar
\sphinxcode{\sphinxupquote{squeeze}}({[}axis{]})
&
\sphinxAtStartPar
Remove axes of length one from \sphinxtitleref{a}.
\\
\sphinxhline
\sphinxAtStartPar
\sphinxcode{\sphinxupquote{std}}({[}axis, dtype, out, ddof, keepdims, where{]})
&
\sphinxAtStartPar
Returns the standard deviation of the array elements along given axis.
\\
\sphinxhline
\sphinxAtStartPar
\sphinxcode{\sphinxupquote{sum}}({[}axis, dtype, out, keepdims, initial, where{]})
&
\sphinxAtStartPar
Return the sum of the array elements over the given axis.
\\
\sphinxhline
\sphinxAtStartPar
\sphinxcode{\sphinxupquote{swapaxes}}(axis1, axis2)
&
\sphinxAtStartPar
Return a view of the array with \sphinxtitleref{axis1} and \sphinxtitleref{axis2} interchanged.
\\
\sphinxhline
\sphinxAtStartPar
\sphinxcode{\sphinxupquote{take}}(indices{[}, axis, out, mode{]})
&
\sphinxAtStartPar
Return an array formed from the elements of \sphinxtitleref{a} at the given indices.
\\
\sphinxhline
\sphinxAtStartPar
\sphinxcode{\sphinxupquote{tobytes}}({[}order{]})
&
\sphinxAtStartPar
Construct Python bytes containing the raw data bytes in the array.
\\
\sphinxhline
\sphinxAtStartPar
\sphinxcode{\sphinxupquote{tofile}}(fid{[}, sep, format{]})
&
\sphinxAtStartPar
Write array to a file as text or binary (default).
\\
\sphinxhline
\sphinxAtStartPar
\sphinxcode{\sphinxupquote{tolist}}()
&
\sphinxAtStartPar
Return the array as an \sphinxcode{\sphinxupquote{a.ndim}}\sphinxhyphen{}levels deep nested list of Python scalars.
\\
\sphinxhline
\sphinxAtStartPar
\sphinxcode{\sphinxupquote{tostring}}({[}order{]})
&
\sphinxAtStartPar
A compatibility alias for \sphinxtitleref{tobytes}, with exactly the same behavior.
\\
\sphinxhline
\sphinxAtStartPar
\sphinxcode{\sphinxupquote{trace}}({[}offset, axis1, axis2, dtype, out{]})
&
\sphinxAtStartPar
Return the sum along diagonals of the array.
\\
\sphinxhline
\sphinxAtStartPar
\sphinxcode{\sphinxupquote{transpose}}(*axes)
&
\sphinxAtStartPar
Returns a view of the array with axes transposed.
\\
\sphinxhline
\sphinxAtStartPar
\sphinxcode{\sphinxupquote{var}}({[}axis, dtype, out, ddof, keepdims, where{]})
&
\sphinxAtStartPar
Returns the variance of the array elements, along given axis.
\\
\sphinxhline
\sphinxAtStartPar
\sphinxcode{\sphinxupquote{view}}({[}dtype{]}{[}, type{]})
&
\sphinxAtStartPar
New view of array with the same data.
\\
\sphinxbottomrule
\end{longtable}
\sphinxtableafterendhook
\sphinxatlongtableend
\end{savenotes}


\begin{savenotes}\sphinxattablestart
\sphinxthistablewithglobalstyle
\centering
\begin{tabulary}{\linewidth}[t]{TT}
\sphinxtoprule
\sphinxtableatstartofbodyhook
\sphinxAtStartPar
\sphinxstylestrong{dot}
&\\
\sphinxbottomrule
\end{tabulary}
\sphinxtableafterendhook\par
\sphinxattableend\end{savenotes}

\end{fulllineitems}



\subparagraph{Module contents}
\label{\detokenize{CoolDwarf.utils.misc:module-CoolDwarf.utils.misc}}\label{\detokenize{CoolDwarf.utils.misc:module-contents}}\index{module@\spxentry{module}!CoolDwarf.utils.misc@\spxentry{CoolDwarf.utils.misc}}\index{CoolDwarf.utils.misc@\spxentry{CoolDwarf.utils.misc}!module@\spxentry{module}}
\sphinxstepscope


\subparagraph{CoolDwarf.utils.plot package}
\label{\detokenize{CoolDwarf.utils.plot:cooldwarf-utils-plot-package}}\label{\detokenize{CoolDwarf.utils.plot::doc}}

\subparagraph{Submodules}
\label{\detokenize{CoolDwarf.utils.plot:submodules}}

\subparagraph{CoolDwarf.utils.plot.plot2d module}
\label{\detokenize{CoolDwarf.utils.plot:module-CoolDwarf.utils.plot.plot2d}}\label{\detokenize{CoolDwarf.utils.plot:cooldwarf-utils-plot-plot2d-module}}\index{module@\spxentry{module}!CoolDwarf.utils.plot.plot2d@\spxentry{CoolDwarf.utils.plot.plot2d}}\index{CoolDwarf.utils.plot.plot2d@\spxentry{CoolDwarf.utils.plot.plot2d}!module@\spxentry{module}}\index{plot\_polar\_slice() (in module CoolDwarf.utils.plot.plot2d)@\spxentry{plot\_polar\_slice()}\spxextra{in module CoolDwarf.utils.plot.plot2d}}

\begin{fulllineitems}
\phantomsection\label{\detokenize{CoolDwarf.utils.plot:CoolDwarf.utils.plot.plot2d.plot_polar_slice}}
\pysigstartsignatures
\pysiglinewithargsret{\sphinxcode{\sphinxupquote{CoolDwarf.utils.plot.plot2d.}}\sphinxbfcode{\sphinxupquote{plot\_polar\_slice}}}{\sphinxparam{\DUrole{n}{sphere}}\sphinxparamcomma \sphinxparam{\DUrole{n}{data}}\sphinxparamcomma \sphinxparam{\DUrole{n}{phi\_slice}\DUrole{o}{=}\DUrole{default_value}{0}}\sphinxparamcomma \sphinxparam{\DUrole{n}{theta\_offset}\DUrole{o}{=}\DUrole{default_value}{0}}\sphinxparamcomma \sphinxparam{\DUrole{n}{fname}\DUrole{o}{=}\DUrole{default_value}{\textquotesingle{}polar\_slice.png\textquotesingle{}}}}{}
\pysigstopsignatures
\end{fulllineitems}

\index{visualize\_scalar\_field() (in module CoolDwarf.utils.plot.plot2d)@\spxentry{visualize\_scalar\_field()}\spxextra{in module CoolDwarf.utils.plot.plot2d}}

\begin{fulllineitems}
\phantomsection\label{\detokenize{CoolDwarf.utils.plot:CoolDwarf.utils.plot.plot2d.visualize_scalar_field}}
\pysigstartsignatures
\pysiglinewithargsret{\sphinxcode{\sphinxupquote{CoolDwarf.utils.plot.plot2d.}}\sphinxbfcode{\sphinxupquote{visualize\_scalar\_field}}}{\sphinxparam{\DUrole{n}{scalar\_field}}\sphinxparamcomma \sphinxparam{\DUrole{n}{slice\_axis}\DUrole{o}{=}\DUrole{default_value}{\textquotesingle{}z\textquotesingle{}}}\sphinxparamcomma \sphinxparam{\DUrole{n}{slice\_index}\DUrole{o}{=}\DUrole{default_value}{None}}\sphinxparamcomma \sphinxparam{\DUrole{o}{**}\DUrole{n}{kwargs}}}{}
\pysigstopsignatures
\end{fulllineitems}



\subparagraph{CoolDwarf.utils.plot.plot3d module}
\label{\detokenize{CoolDwarf.utils.plot:module-CoolDwarf.utils.plot.plot3d}}\label{\detokenize{CoolDwarf.utils.plot:cooldwarf-utils-plot-plot3d-module}}\index{module@\spxentry{module}!CoolDwarf.utils.plot.plot3d@\spxentry{CoolDwarf.utils.plot.plot3d}}\index{CoolDwarf.utils.plot.plot3d@\spxentry{CoolDwarf.utils.plot.plot3d}!module@\spxentry{module}}\index{plot\_3d\_gradients() (in module CoolDwarf.utils.plot.plot3d)@\spxentry{plot\_3d\_gradients()}\spxextra{in module CoolDwarf.utils.plot.plot3d}}

\begin{fulllineitems}
\phantomsection\label{\detokenize{CoolDwarf.utils.plot:CoolDwarf.utils.plot.plot3d.plot_3d_gradients}}
\pysigstartsignatures
\pysiglinewithargsret{\sphinxcode{\sphinxupquote{CoolDwarf.utils.plot.plot3d.}}\sphinxbfcode{\sphinxupquote{plot\_3d\_gradients}}}{\sphinxparam{\DUrole{n}{grid\_points}}\sphinxparamcomma \sphinxparam{\DUrole{n}{gradients}}\sphinxparamcomma \sphinxparam{\DUrole{n}{radius}}\sphinxparamcomma \sphinxparam{\DUrole{n}{sphere\_radius}\DUrole{o}{=}\DUrole{default_value}{1}}\sphinxparamcomma \sphinxparam{\DUrole{n}{cell}\DUrole{o}{=}\DUrole{default_value}{False}}}{}
\pysigstopsignatures
\end{fulllineitems}



\subparagraph{Module contents}
\label{\detokenize{CoolDwarf.utils.plot:module-CoolDwarf.utils.plot}}\label{\detokenize{CoolDwarf.utils.plot:module-contents}}\index{module@\spxentry{module}!CoolDwarf.utils.plot@\spxentry{CoolDwarf.utils.plot}}\index{CoolDwarf.utils.plot@\spxentry{CoolDwarf.utils.plot}!module@\spxentry{module}}

\paragraph{Module contents}
\label{\detokenize{CoolDwarf.utils:module-CoolDwarf.utils}}\label{\detokenize{CoolDwarf.utils:module-contents}}\index{module@\spxentry{module}!CoolDwarf.utils@\spxentry{CoolDwarf.utils}}\index{CoolDwarf.utils@\spxentry{CoolDwarf.utils}!module@\spxentry{module}}

\subsection{Module contents}
\label{\detokenize{CoolDwarf:module-CoolDwarf}}\label{\detokenize{CoolDwarf:module-contents}}\index{module@\spxentry{module}!CoolDwarf@\spxentry{CoolDwarf}}\index{CoolDwarf@\spxentry{CoolDwarf}!module@\spxentry{module}}

\section{Indices and tables}
\label{\detokenize{index:indices-and-tables}}\begin{itemize}
\item {} 
\sphinxAtStartPar
\DUrole{xref,std,std-ref}{genindex}

\item {} 
\sphinxAtStartPar
\DUrole{xref,std,std-ref}{modindex}

\item {} 
\sphinxAtStartPar
\DUrole{xref,std,std-ref}{search}

\end{itemize}


\renewcommand{\indexname}{Python Module Index}
\begin{sphinxtheindex}
\let\bigletter\sphinxstyleindexlettergroup
\bigletter{c}
\item\relax\sphinxstyleindexentry{CoolDwarf}\sphinxstyleindexpageref{CoolDwarf:\detokenize{module-CoolDwarf}}
\item\relax\sphinxstyleindexentry{CoolDwarf.EOS}\sphinxstyleindexpageref{CoolDwarf.EOS:\detokenize{module-CoolDwarf.EOS}}
\item\relax\sphinxstyleindexentry{CoolDwarf.EOS.ChabrierDebras2021}\sphinxstyleindexpageref{CoolDwarf.EOS.ChabrierDebras2021:\detokenize{module-CoolDwarf.EOS.ChabrierDebras2021}}
\item\relax\sphinxstyleindexentry{CoolDwarf.EOS.ChabrierDebras2021.EOS}\sphinxstyleindexpageref{CoolDwarf.EOS.ChabrierDebras2021:\detokenize{module-CoolDwarf.EOS.ChabrierDebras2021.EOS}}
\item\relax\sphinxstyleindexentry{CoolDwarf.EOS.EOS}\sphinxstyleindexpageref{CoolDwarf.EOS:\detokenize{module-CoolDwarf.EOS.EOS}}
\item\relax\sphinxstyleindexentry{CoolDwarf.EOS.invert}\sphinxstyleindexpageref{CoolDwarf.EOS.invert:\detokenize{module-CoolDwarf.EOS.invert}}
\item\relax\sphinxstyleindexentry{CoolDwarf.EOS.invert.EOSInverter}\sphinxstyleindexpageref{CoolDwarf.EOS.invert:\detokenize{module-CoolDwarf.EOS.invert.EOSInverter}}
\item\relax\sphinxstyleindexentry{CoolDwarf.err}\sphinxstyleindexpageref{CoolDwarf.err:\detokenize{module-CoolDwarf.err}}
\item\relax\sphinxstyleindexentry{CoolDwarf.err.energy}\sphinxstyleindexpageref{CoolDwarf.err:\detokenize{module-CoolDwarf.err.energy}}
\item\relax\sphinxstyleindexentry{CoolDwarf.err.eos}\sphinxstyleindexpageref{CoolDwarf.err:\detokenize{module-CoolDwarf.err.eos}}
\item\relax\sphinxstyleindexentry{CoolDwarf.ext}\sphinxstyleindexpageref{CoolDwarf.ext:\detokenize{module-CoolDwarf.ext}}
\item\relax\sphinxstyleindexentry{CoolDwarf.model}\sphinxstyleindexpageref{CoolDwarf.model:\detokenize{module-CoolDwarf.model}}
\item\relax\sphinxstyleindexentry{CoolDwarf.model.dsep}\sphinxstyleindexpageref{CoolDwarf.model.dsep:\detokenize{module-CoolDwarf.model.dsep}}
\item\relax\sphinxstyleindexentry{CoolDwarf.model.dsep.dsep}\sphinxstyleindexpageref{CoolDwarf.model.dsep:\detokenize{module-CoolDwarf.model.dsep.dsep}}
\item\relax\sphinxstyleindexentry{CoolDwarf.model.mesa}\sphinxstyleindexpageref{CoolDwarf.model.mesa:\detokenize{module-CoolDwarf.model.mesa}}
\item\relax\sphinxstyleindexentry{CoolDwarf.model.mesa.mesa}\sphinxstyleindexpageref{CoolDwarf.model.mesa:\detokenize{module-CoolDwarf.model.mesa.mesa}}
\item\relax\sphinxstyleindexentry{CoolDwarf.model.model}\sphinxstyleindexpageref{CoolDwarf.model:\detokenize{module-CoolDwarf.model.model}}
\item\relax\sphinxstyleindexentry{CoolDwarf.opac}\sphinxstyleindexpageref{CoolDwarf.opac:\detokenize{module-CoolDwarf.opac}}
\item\relax\sphinxstyleindexentry{CoolDwarf.opac.aesopus}\sphinxstyleindexpageref{CoolDwarf.opac.aesopus:\detokenize{module-CoolDwarf.opac.aesopus}}
\item\relax\sphinxstyleindexentry{CoolDwarf.opac.aesopus.load}\sphinxstyleindexpageref{CoolDwarf.opac.aesopus:\detokenize{module-CoolDwarf.opac.aesopus.load}}
\item\relax\sphinxstyleindexentry{CoolDwarf.opac.kramer}\sphinxstyleindexpageref{CoolDwarf.opac:\detokenize{module-CoolDwarf.opac.kramer}}
\item\relax\sphinxstyleindexentry{CoolDwarf.opac.opacInterp}\sphinxstyleindexpageref{CoolDwarf.opac:\detokenize{module-CoolDwarf.opac.opacInterp}}
\item\relax\sphinxstyleindexentry{CoolDwarf.star}\sphinxstyleindexpageref{CoolDwarf.star:\detokenize{module-CoolDwarf.star}}
\item\relax\sphinxstyleindexentry{CoolDwarf.star.sphere}\sphinxstyleindexpageref{CoolDwarf.star:\detokenize{module-CoolDwarf.star.sphere}}
\item\relax\sphinxstyleindexentry{CoolDwarf.utils}\sphinxstyleindexpageref{CoolDwarf.utils:\detokenize{module-CoolDwarf.utils}}
\item\relax\sphinxstyleindexentry{CoolDwarf.utils.const}\sphinxstyleindexpageref{CoolDwarf.utils.const:\detokenize{module-CoolDwarf.utils.const}}
\item\relax\sphinxstyleindexentry{CoolDwarf.utils.const.const}\sphinxstyleindexpageref{CoolDwarf.utils.const:\detokenize{module-CoolDwarf.utils.const.const}}
\item\relax\sphinxstyleindexentry{CoolDwarf.utils.format}\sphinxstyleindexpageref{CoolDwarf.utils.format:\detokenize{module-CoolDwarf.utils.format}}
\item\relax\sphinxstyleindexentry{CoolDwarf.utils.format.format}\sphinxstyleindexpageref{CoolDwarf.utils.format:\detokenize{module-CoolDwarf.utils.format.format}}
\item\relax\sphinxstyleindexentry{CoolDwarf.utils.interp}\sphinxstyleindexpageref{CoolDwarf.utils.interp:\detokenize{module-CoolDwarf.utils.interp}}
\item\relax\sphinxstyleindexentry{CoolDwarf.utils.interp.interpolate}\sphinxstyleindexpageref{CoolDwarf.utils.interp:\detokenize{module-CoolDwarf.utils.interp.interpolate}}
\item\relax\sphinxstyleindexentry{CoolDwarf.utils.math}\sphinxstyleindexpageref{CoolDwarf.utils.math:\detokenize{module-CoolDwarf.utils.math}}
\item\relax\sphinxstyleindexentry{CoolDwarf.utils.math.calc}\sphinxstyleindexpageref{CoolDwarf.utils.math:\detokenize{module-CoolDwarf.utils.math.calc}}
\item\relax\sphinxstyleindexentry{CoolDwarf.utils.math.kernel}\sphinxstyleindexpageref{CoolDwarf.utils.math:\detokenize{module-CoolDwarf.utils.math.kernel}}
\item\relax\sphinxstyleindexentry{CoolDwarf.utils.misc}\sphinxstyleindexpageref{CoolDwarf.utils.misc:\detokenize{module-CoolDwarf.utils.misc}}
\item\relax\sphinxstyleindexentry{CoolDwarf.utils.misc.logging}\sphinxstyleindexpageref{CoolDwarf.utils.misc:\detokenize{module-CoolDwarf.utils.misc.logging}}
\item\relax\sphinxstyleindexentry{CoolDwarf.utils.misc.ndarray}\sphinxstyleindexpageref{CoolDwarf.utils.misc:\detokenize{module-CoolDwarf.utils.misc.ndarray}}
\item\relax\sphinxstyleindexentry{CoolDwarf.utils.plot}\sphinxstyleindexpageref{CoolDwarf.utils.plot:\detokenize{module-CoolDwarf.utils.plot}}
\item\relax\sphinxstyleindexentry{CoolDwarf.utils.plot.plot2d}\sphinxstyleindexpageref{CoolDwarf.utils.plot:\detokenize{module-CoolDwarf.utils.plot.plot2d}}
\item\relax\sphinxstyleindexentry{CoolDwarf.utils.plot.plot3d}\sphinxstyleindexpageref{CoolDwarf.utils.plot:\detokenize{module-CoolDwarf.utils.plot.plot3d}}
\end{sphinxtheindex}

\renewcommand{\indexname}{Index}
\printindex
\end{document}